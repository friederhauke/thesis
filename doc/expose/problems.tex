With the implementation but also with the theoretical questions several problems could occur.\\
First of all the biggest problem would be if there are no data for any testing available. Either due to the fact that there is not an appropriate project taking place or because of data-rights problems.\\
The possibility of a cooperation with the 'IGG Students Project' of the winter term 2013/14 might be a proper data source.\\
If there is a fitting project willing to provide its data, the used sensors might not be able to connect to the Web, and therewith one important part of the developed System cannot be tested of even worse not been developed.\\
For the streaming aspect of the System it is important that the used data are small enough for a time efficient sending via the web. In case of e.g. raw-data from TLS this would not be the case. The usage of preprocessing the data in the field and sending only intermediate results (in form of e.g. gml-geometries) might solving this problem, but errors in the necessary algorithms are invoking new problems.\\
Since the full system is depending on a database which should be given by previous work, one big problem could occur when the design of the database is inefficient for the planned tasks, or when the design is covering the full bandwidth of data the System is going to handle.\\
In the end the system will run on mobile devices. Up to now there have been three platforms established on the Smartphone and Tablet-PC marked: Googles Android, Apples iOS and Microsofts Windows. The development in context of this master thesis will at the most consider only on of these operating systems as a testing platform. The problem here would be a incompatibility to some existing platforms.