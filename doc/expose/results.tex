As an overall output an demonstrating prototype of the system will be developed based on one concrete example. Most open questions will in the end be answered and evaluated, based on the developed prototype.\\
The demonstrating example will contain data from different sensors like Levelling Instruments, Total Stations and possibly also data from Accelerometers and other permanent observation sensors or Terrestrial Laser Scanners.\\
The data will be transferred to a central database in such a quality and short interval that a direct and continuous observation of changes in the measurements on a mobile device can be done.\\
On the structural health monitoring test side for this prototype, experiments should be part of the evaluation of the developed system. Such experiments like observation of loading test on the monitored construction (e.g. a bridge). \\
Changes in the measured data should be illustrated in a practicable way. An interpretation of numeric results by the user is not an option. Better would be a graphical symbolisation of changes and it pre-evaluation. What kind and which statistical interpretation of the measurements will be part of the algorithms will not be part of the development of this prototype and should be offered by the partnered scientists or parallel master thesis.\\
This master thesis will simply develop the technological structure as a basis for any implementation of adjustment and statistical evaluation algorithms.\\
