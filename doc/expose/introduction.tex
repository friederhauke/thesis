This master thesis will first describe potential user needs in a graphical management and analysis tool for the structural health system and secondly develop a system which will meet the described needs. This system will help field engineers generally in:
\begin{itemize}
\item planning and optimising their field work,
\item corresponding with the office,
\item and quickly evaluate measured values.
\end{itemize}
Classical work in survey engineering consist of the measuring tasks in the field, and its evaluation, analysis and correction in the office. This subdivision into two spatially different, but often very cooperating tasks leads to inefficiency of work, time delay and avoidable mistakes in the work flows.\\
For the sake of completeness, however, it should also be pointed out that not all working tasks of field engineers can be done in the field, therewith the system is limited to this existing frame.\\
Nowadays a close interaction of different activities every time and everywhere is not any more a vision, it is common to our society and people get used to it. Field work today already contains usages of mobile devices which help organising the work flow.\\
There are three main advantages of using such a system following this approach:
\begin{itemize}
\item The Systems core functionality will be the assistance of the survey engineer. This means it helps to get a better 'understanding' of the measurements e.g. in comparison to former epochs. $\rightarrow$ The sharing of information between field and office leads to a better knowledge during field work. It should support the decisions in field.
\item It is avoiding "mis-measurements" caused due to a incomplete knowledge of the measurement network by validation of measurement "on the fly" directly after the upload to the systems database.
\item For a field exploration existing results of the former measures have not to be exported from the office infrastructure, the mobile system is linking real time data with existing results.
\end{itemize}
There is a need of such a system, the combination of used techniques like "Sensor Web Enablement", OGCs Web Services and mobile advanced IT platforms like tablet computers for a streaming of data from the field and to the field is up to now an unanswered question. This master thesis will explore possible solutions and the real need of potential users.\\
It will not be an alternative for engineering management systems which are already running on mobile devices, the real benefit of this system being designed with this thesis will be a direct linkage of the sensors to the systems, from data streaming up to augmented reality. The combination of classical surveying methods with new sensor observations in case of structural health monitoring will open new possibilities and new perspectives for field work of survey engineers\\
