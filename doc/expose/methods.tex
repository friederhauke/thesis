Based on existing developments like IT-Architectures and Data-formats the system will be designed by the actual need of working field engineers. Use case models and research on already existing systems will help in finding a technically optimal solution which should be also economically feasible.\\
Main questions on the concept will be:
\begin{itemize}
\item Which Sensors will be linked, and what kind of data will be used (in case of e.g. interval)?
\item How will the data be transferred (from the sensors to the office and to the mobile device), and where will they be stored?
\item How will the graphical user interface be structured and designed?
\item What kind of IT-architecture and which protocols will be used?
\item Is the already existing database covering all necessary issues?
\end{itemize}
To find a working and operationally usable solution following questions, problems and concepts have to be reviewed, discussed and evaluated:\\
\subsection{Data Insertion}
When talking about a structural health monitoring system, the data of different kind of sensors are part of the evaluation calculations. Therewith the System has to deal with different ways of data input. There can be automatically harvested data from sensors like accelerometers or GPS-Stations (Global Positioning System), but also manual insertion of data from sensors like levelling instruments or additional control measurements should be possible. This is of course also a task for the database: how to deal with different kind of features.\\
Occurring questions like how to deal with different intervals (e.g. accelerometers are sending data each second while levelings are inserted only e.g. once a month) have to be answered.\\
But also the question how to insert manual changes of the data in field by the engineer (e.g. on a tablet computer) into the database has to be answered.
\subsection{Complex Data}
The used database then has to be able to store those different kind of data: Starting with point clouds from terrestrial laser scanners up to feature models of the observed structure. And also the raw material might be important for future scientific work, therewith maybe also pictures have to be stored. Projects like the GITEWS (German Indonesian Tsunami Early Warning System have already developed examples for databases storing complex data \citep{strobl_geodatenmanagement_2007}.
\subsection{Time Series}
Second problem in case of the database will be the storage of time-series of the data. Monitoring means changes of values in comparison to former epochs. In our case most changing values are spatial information, and therewith the problem describes the storage of moving objects in a database.\\
Several approaches exist dealing with this problem, first: For every change of one parameter a full snapshot of all data will be stored in the database but this would lead to a big amount of data.\\
Second: Only the changed parameter will be stored, but then the relationship model of the database entities and therewith the managing system will become very complex. \citep{erwig_spatio-temporal_1999}\citep{koubarakis_spatio-temporal_2003}\citep{yuan_temporal_1996}
\subsection{Data Streaming and Visualisation}
The System then has to be able to stream and visualise the data on a certain platform. Webservices might be the actually best solution and they fit to the trend offering software as a service. The service oriented architecture paradigm described in \citep{papazoglou_web_2008} describes the method how to serve the relevant information without having a monolithic static system. With geometric data, as for example geometric primitives, the encoding of the data in CityGML \citep{gerhard_groger_ogc_2012}\citep{kolbe_3d-geo-database_2009} would be the best solution. With the help of Services defined by the Open Geospatial Consortium (OGC) like for example the Web Feature Service \citep{panagiotis_a._vretanos_opengis_2005} \citep{jeff_de_la_beaujardiere_opengis_2006}\citep{douglas_nebert_opengis_2007}\citep{panagiotis_a._vretanos_opengis_2010}\citep{kolbe_draft_2009} the related data could be visualised.\\
Central questions will be what will be visualised how and with which technology. What makes sense and helps the engineer in field with his work. Which kind of data representation and which dimension makes sense in order to symbolise erroneous measurements and or changes in the results.
\subsection{Data Analysis}
Not as a central part of this master thesis but at least partially the analysis of the used data should be described. The development of appropriate algorithms might not be a task, but the way how to integrate them into the structure.\\
The Finite Element Method (FEM see e.g. \citep{zienkiewicz_finite_1977}) should be mentioned here as an example relevant analysis tool.
\subsection{Feature Extraction}
The System has to be able to access the Sensors and to translate the data in to machine readable and understandable information. In a very complex case the data of e.g. a Terrestrial Laser Scanner is a point cloud which is unique in each epoch. Therewith the relevant absolute information have to be extracted, mostly in this case geometrical features. The RANSAC (RANdom SAmple Consensus) approach has been turned out as one feasible solution for the extraction of geometric Features \citep{schnabel_efficient_2007}. And for the recognition of semantic features there are existing different approaches e.g. \citep{schnabel_efficient_2010}\citep{gumhold_feature_2001}.\\
But also the more simple sensor types are providing data which have to be translated into machine understandable information. E.g. the accelerometer is producing numerical values without any relation to geographic position or sensor orientation. Here the system has to find a way how to harmonise all the different data inputs.
\subsection{Testside}
In cooperation with a students project at the Institute of Geodesy and Geoinformation Science running in parallel during the winter term 2013/14 a bridge model as a test side for structural health monitoring and Sensor Web Enablement can be used also for testings of the system being developed within this master thesis.\\
It has to be discovered how to access the sensors and whether it is covering all planned features of the system.