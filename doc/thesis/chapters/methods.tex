\chapter{Werkzeuge}

\section{Bauwerksüberwachung}
In der Einleitung wurde bereits beschrieben, wozu dieses System dienen soll, und was unter der Überwachung von Bauwerken zu verstehen ist. 

\subsection{Sensortypen}
\subsubsection{manuelle Sensoren}
\subsubsection{automatische Sensoren}
\subsection{Relevante Eigenschaften}

\section{Entscheidungs-Unterstützung}

\section{Sensornetzwerke}
Diese Arbeit beschreibt ein System, das die Arbeit mit einem Netzwerk aus Sensoren organisieren und vereinfachen soll. Werden Sensoren in einem Netzwerk organisiert, übernimmt die Aufgabe des Messens eine Software die den Sensor fernsteuert. Klassischerweise gehört dazu das Messen selbst, das Speichern der Daten und gegebenenfalls die Auswertung der Daten und damit verbunden eine mögliche Ereigniserkennung. Arbeitsabläufe die für diese Arbeit von Belang sind wurden in den Anforderungen zu Beginn der Arbeit detailliert beschrieben. 

Sensornetzwerke spielen bei einer Großzahl der im Alltag integrierten technischen Geräte bereits eine große Rolle. In modernen Autos werden kontinuierlich Parameter wie \"Spritverbrauch\" oder Drehzahl gemessen und dann eine mögliche verbleibende Reichweite errechnet. In Computern drehen die Lüfter erst dann in einer höheren Drehzahl wenn die automatische gemessene Temperatur des Prozessors eine festgelegte stufe erreicht. Und in besonders sensiblen Geräten wie Flugzeugen findet sich eine noch viel größere Anzahl von Sensoren und auch die Auswertung der gelieferten Messerwerte wird durch komplexe Algorithmen im Cockpit für den Piloten verständlich angezeigt. Die hier aufgezählten Beispiele stellen aber jeweils ein in sich geschlossenes System dar, bei dem Datenübermittlung oder Datenmanagement "einfach" geregelt werden kann. "Einfach" soll aber nichts als technische trivial verstanden werden, sondern lediglich darauf hinweisen, dass die dafür verwendete Technologie bereits sehr weit entwickelt ist. 

In Bereichen deren wissenschaftliche Ausrichtung die Überwachung von räumlich größeren Objekten ist, halten seit einigen Jahren verstärkt Sensornetzwerke Einzug (siehe hier auch die Vorstellung der Projekte in der Einleitung). Derartige Sensornetzwerke haben mit einer Vielzahl von Problemen zu kämpfen, die gerade in der großen räumliche Ausdehnung begründet ist. Dazu gehören die Stromversorgung im Feld, die Datenübertragung über große Strecken und nicht zuletzt häufig auch die Witterung welche Einfluss auf verschiedene Faktoren nimmt. Um diese Probleme zu lösen haben sich grundsätzlich zwei verschiedene Architekturen von Sensornetzwerken herausgebildet. Zum Einen agieren die Sensoren autonom, zum anderen sind sie Abhängig von einem zentralen Knoten. Ich werde beide Ansätze im Folgenden näher erläutern und gegenüberstellen.

\subsection{autonome Sensornetzwerke}
Sensoren in autonomen Netzwerken stellen jeweils für sich einen autonomen Knoten dar. Die Besonderheit ist, dass nicht ein zentraler Knoten alle anderen Organisiert, sondern dass sich das Netzwerk selber organisiert, indem die sich Knoten aktiv einbringen, und Daten senden oder empfangen. Konfigurieren sich die Knoten auch selber spricht man auch von einem sogenannten "Ad-Hoc Netzwerk" \citep{voigt_illustration_2013}. Autonome Sensornetzwerke wurden ursprünglich für militärische Zwecke entwickelt, werde heutzutage aber auf vielfältige Art im zivilen Bereichen eingesetzt.

Einen Vorteil, den autonome Sensornetzwerke mit sich bringen, ist die schnelle und bequeme Installation. Dadurch, dass viel Vorarbeit in die Entwicklung der Software und Hardware der Sensoren gesteckt wurde, kommt die Installation ohne manuelle Konfiguration und damit hohem Zeitaufwand aus. Von Nachteil ist allerdings, dass durch den Betrieb von zusätzlicher Hardware die aus dem Sensor einen vollwertigen Knoten machen, der Energieverbrauch meist um ein Vielfaches über dem liegt, was der Sensor alleine benötigt.

Die Knoten bei autonomen Sensornetzwerken bestehen meist aus einem oder mehreren Sensoren, einer Recheneinheit sowie der Kommunikationseinheit. Je nachdem ob die Netzwerke drahtlos oder verkabelt betrieben werden, oder ob sie einen Stromanschluss besitzen, kommen zudem noch Komponenten wie einer Antenne, oder einer Batterie dazu. Die aktuelle Forschung konzentriert sich besonders auf diese kritischen Teile der Sensorknoten. Durch den Betrieb vieler Komponenten und besonders einer drahtlosen Kommunikation wird viel Strom verbraucht, der die Operationszeit stark verkürzt. Zudem kommt, dass derart komplexe Knoten momentan noch zu teuer sind, um sie in einer größeren Menge, wie sie vielleicht erforderlich wäre, installieren zu können.

Benjamin Voigt, siehe \citep{voigt_illustration_2013}, beschreibt in seinem Artikel unter anderem ein drahtloses Sensornetzwerk auf einem Vulkan, dessen Zuverlässigkeit in Bezug auf die Erkennung von Ereignissen in einem zeitlich abgegrenztem Experiment untersucht wurde. Alle relevanten Ereignisse wurden zwar zuverlässig erkannt, darüber hinaus aber 99\%  nicht relevante Ereignisse. Zu den schlechten Ergebnissen führten vor Allem Fehler der Software und der Algorithmen. Das lässt den Autoren die Schlussfolgerung ziehen, dass solch ein System noch nicht für den operativen Dienst verwendbar ist. Hauptsächlich auch deshalb, weil die Lösungsansätze (höherer Betreuungsaufwand) bei Netzwerken mit einer realistischen Anzahl von Knoten (<1000) nicht finanzierbar wären.

\subsection{nicht-autonome Sensornetzwerke}


\subsection{IT Konzepte}
\subsubsection{SensorML}
\subsubsection{Metadaten}
\subsubsection{Sensor Web Enablement}
\paragraph{52North}
\paragraph{istSOS}

\section{Dateninfrastruktur}
\subsection{Datentypen und Datenformate}
\subsubsection{Keyhole Markup Language}
\subsubsection{Geographic Markup Language}
\subsection{Geodatenbanken}
\subsubsection{PostGIS}
\subsubsection{Oracle Spatial}
\subsection{Webdienste}
\subsubsection{Web Map Service}
\subsubsection{Web Feature Service}
\subsubsection{Web Catalogue Service}

\section{Analysemethoden}
\subsection{Methode der finiten Elemente}
\subsection{Datenkonsistenz}

\section{Mobile Dienste}
In der Einleitung bin ich bereits umfangreich auf die besonderen Charakteristiken von mobilen Diensten eingegangen, insbesondere auf die nicht-funktionellen Eigenschaften. In diesem Kapitel skizziere ich die aktuellen Forschungen in diesem Bereich und gehe detailliert darauf ein welche Rolle sie für diese Arbeit spielen können.

Der bereits zitierte Martin Breunig, siehe \citep{breunig_entwicklung_2003}, wies bereits früh auf das Potential von Geographischen Informationen für mobile Dienste hin. Die wesentlichen Teile eines solchen Systems haben sich bis heute kaum verändert, deren heutiger Entwicklungsstand ist jedoch seit dem um einiges weiter fortgeschritten. Breunig beschreibt ebenfalls die Nutzung von standardisierten Schnittstellen (siehe Kapitel Dateninfrastruktur), die besonderen Anforderungen der System-Architektur sowie die Datenvisualisierung als Kernpunkte eines Konzeptes für ein mobiles System.


\chapter{Softwareentwurf}
