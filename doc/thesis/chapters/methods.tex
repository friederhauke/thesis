\chapter{Werkzeuge}
In den nachfolgenden Kapiteln gebe ich detaillierte Informationen zu den möglichen und alternativen Bausteinen für das geplante System. Die Informationen sollen der besseren Verständlichkeit des Softwaredesigns und der verwendeten System-Architektur dienen. Da mein Lösungsansatz nur ein Möglicher und keinesfalls der Einzige ist, dienen die Informationen auch dazu die Funktionsweise des Prototypen zu verstehen und kritisch zu beurteilen.

\section{Bauwerksüberwachung}
In der Einleitung wurde bereits beschrieben, wozu dieses System dienen soll, und was unter der Überwachung von Bauwerken zu verstehen ist. Anders als bei den bereits weit verbreiteten Sensortechniken, die der Überwachung von relativ kleinen technischen Objekten dienen, müssen bei der Überwachung von Gebäuden die Sensoren an teilweise sehr schwer zugänglichen und voneinander weit entfernten Positionen angebracht werden. Die Wahl einer Technologie die für die Kommunikation die bereits bestehende Infrastruktur des Internets nutzen kann umgeht das Problem komplizierter Verkabelungen.

\subsection{Sensortypen}
\subsubsection{manuelle Sensoren}
\subsubsection{automatische Sensoren}
\subsection{Relevante Eigenschaften}

\section{Entscheidungs-Unterstützung}

\section{Sensornetzwerke}
Diese Arbeit beschreibt ein System, das die Arbeit mit einem Netzwerk aus Sensoren organisieren und vereinfachen soll. Sensornetzwerke sind räumlich verteilte Sensoren die miteinander verbunden sind und über einen Computer gesteuert werden können \citep{botts_ogc_2008}. Der Grund warum Sensoren in einem Netzwerk organisiert werden sollten liegt darin, dass so Datenströme gebündelt werden können, die Messdaten durch eine Homogenisierung der Datentypen vergleichbar gemacht werden und der Wartungsaufwand für das Komplettsystem verringert wird, siehe auch \citep{resch_standardisierte_2012} \citep{bermudez_ogc_2011}. Werden Sensoren in einem Netzwerk organisiert, übernimmt die Aufgabe des Messens eine Software die den Sensor fernsteuert. Klassischerweise gehört dazu das Messen selbst, das Speichern der Daten und gegebenenfalls die Auswertung der Daten und damit verbunden eine mögliche Ereigniserkennung. Arbeitsabläufe, die für diese Arbeit von Belang sind, wurden in den Anforderungen zu Beginn der Arbeit bereits detailliert beschrieben. 

Im Englischen wird oft auch der Begriff des ``Sensor Web'' verwendet, dessen Bedeutung über die des Sensornetzwerkes (englisch ``Sensor Network'' hinaus geht. Um die beiden Begriffe eindeutig voneinander zu unterscheiden werde ich im Folgenden von Sensornetzwerken und von einem Sensor-Web sprechen. Das Sensor-Web besteht aus Sensoren die ebenfalls miteinander verbunden sind, aber auch über das Internet erreichbar sind. Das Sensor-Web stellt somit eine Erweiterung des bloßen Netzwerkes dar, weil über das Internet Sensoren verschiedener Netzwerke miteinander verbunden werden können. Die erfassten Daten der Sensoren werden über das Internet verfügbar gemacht und können über Standard Protokolle oder einer sogenannten \newacronym{API}{API}{Application Programming Interface, deutsch: Programmierschnittstelle} \gls{API} abgerufen werden \citep{broring_new_2011}\citep{botts_ogc_2008}\citep{guinard_towards_2009}.

Das Sensor Web kann als Kommunikationsebene zwischen den Sensoren und der die Daten nutzenden Software gesehen werden. Abbildung \ref{fig:swe_layer-stack} zeigt die einzelnen Ebenen der Architektur und sortiert in dem Zusammenhang häufig genannte Begriffe ein. In der Sensor Ebene ist die eigentliche Sensor Hardware organisiert. Die Kommunikation mit den Nutzern der Systeme findet sich in der Applikations Ebene.

\begin{figure}[H]
	\centering
 	 \includegraphics[scale=0.3]{graphics/SWE_layer-stack.jpg} 
	\caption{Ebenen des Sensor Web und Einsortierung von relevanten Begriffen nach \citep{broring_new_2011}}
	 \label{fig:swe_layer-stack}
\end{figure}

Noch weiter ging Kevin Ashton als er 1999 erstmals den Begriff des Internet der Dinge (englisch: ``Internet of things\") verwendete. In seiner Präsentation 1999 zum Thema \newacronym{RFID}{RFID}{Radio-Frequency Identification} \gls{RFID} bei Procter \& Gamble formulierte er eine These derzufolge in der Zukunft Computer über Sensoren ohne die Mithilfe des Menschen Informationen über die Umwelt sammeln könnten um so den Menschen zu entlasten und ihm mehr Wissen zu verschaffen als er selbst im Stande wäre. Ashton begründete dies mit der Aussage, dass das momentane Internet aus reinen Informationen besteht, Dinge wie Lebensmittel oder Brennstoffe aber wesentlich wichtiger sind, und das Internet somit auch über diese Dinge Bescheid wissen sollte \citep{ashton_that_2009}.

Das ``Web'' der Dinge wiederum ist nach \citep{broring_new_2011} eine Erweiterung des Internet der Dinge, da es gängige Protokolle für die Kommunikation zwischen realen Objekten verwendet. Funktionalitäten realer Objekte werden mittels \gls{API} über das \newacronym{HTTP}{HTTP}{Hypertext Transfer Protocol}\gls{HTTP} im Web verfügbar gemacht \citep{guinard_towards_2009}.

Sensornetzwerke spielen bei einer Großzahl der im Alltag integrierten technischen Geräte bereits eine große Rolle. In modernen Autos werden kontinuierlich Parameter wie ``Spritverbrauch'' oder Drehzahl gemessen und dann eine mögliche verbleibende Reichweite errechnet. In Computern drehen die Lüfter erst dann in einer höheren Drehzahl wenn die automatische gemessene Temperatur des Prozessors eine festgelegte stufe erreicht. Und in besonders sensiblen Geräten wie Flugzeugen findet sich eine noch viel größere Anzahl von Sensoren und auch die Auswertung der gelieferten Messerwerte wird durch komplexe Algorithmen im Cockpit für den Piloten verständlich angezeigt. Die hier aufgezählten Beispiele stellen aber jeweils ein in sich geschlossenes System dar, bei dem Datenübermittlung oder Datenmanagement "einfach" geregelt werden kann. "Einfach" soll aber nichts als technische trivial verstanden werden, sondern lediglich darauf hinweisen, dass die dafür verwendete Technologie bereits sehr weit entwickelt ist. 

Die Knoten bestehen meist aus einem oder mehreren Sensoren, einer Recheneinheit sowie der Kommunikationseinheit. Je nachdem ob die Netzwerke drahtlos oder verkabelt betrieben werden, oder ob sie einen Stromanschluss besitzen, kommen zudem noch Komponenten wie einer Antenne, oder einer Batterie dazu. Die aktuelle Forschung konzentriert sich besonders auf diese kritischen Teile der Sensorknoten. Durch den Betrieb vieler Komponenten und besonders einer drahtlosen Kommunikation wird viel Strom verbraucht, der die Operationszeit stark verkürzt. Zudem kommt, dass derart komplexe Knoten momentan noch zu teuer sind, um sie in einer größeren Menge, wie sie vielleicht erforderlich wäre, installieren zu können. Siehe dazu auch \citep{akyildiz_survey_2002}.

In Bereichen deren wissenschaftliche Ausrichtung die Überwachung von räumlich größeren Objekten ist, halten seit einigen Jahren verstärkt Sensornetzwerke Einzug (siehe hier auch die Vorstellung der Projekte in der Einleitung). Derartige Sensornetzwerke haben mit einer Vielzahl von Problemen zu kämpfen, die gerade in der großen räumliche Ausdehnung begründet ist. Dazu gehören die Stromversorgung im Feld, die Datenübertragung über große Strecken und nicht zuletzt häufig auch die Witterung welche Einfluss auf verschiedene Faktoren nimmt. In der Einleitung wurden bereits zwei Projekte vorgestellt, die sich mit dem Aufbau und dem Betrieb von Sensornetzwerken zur Überwachung des Erdkörpers befassten. Ein Anderer Anwendungsfall für Sensornetzwerke wäre zum Beispiel das Kontrollieren von Überland-Stromleitungen wie es in \citep{voigt_autarkes_2012} beschrieben wird. Probleme wie der Strombedarf oder die Datenübermittlung wurden hierbei durch technisch an die Situation angepasste Eigenentwicklungen gelöst.

Näher eingehen möchte ich auf die Besonderheiten der autonomen Sensor Netzwerke. Sensoren in autonomen Netzwerken stellen jeweils für sich einen autonomen Knoten dar. Ein zentraler Knoten der alle Sensoren Organisiert ist somit nicht notwendig. Vielmehr organisiert sich das Netzwerk selber indem die sich Knoten aktiv einbringen und Daten aktiv senden oder empfangen. Konfigurieren sich die Knoten auch selber spricht man auch von einem sogenannten ``Ad-Hoc Netzwerk''. Autonome Sensornetzwerke wurden ursprünglich für militärische Zwecke entwickelt, werde heutzutage aber auf vielfältige Art im zivilen Bereichen eingesetzt. Die immer komplexeren Anforderungen an die technische Überwachung diverser Phänomene fördert die Entwicklung von autonomen Ad-Hoc Netzwerken. Einen Vorteil, den autonome Sensornetzwerke mit sich bringen, ist die schnelle und bequeme Installation. Dadurch, dass viel Vorarbeit in die Entwicklung der Software und Hardware der Sensoren gesteckt wurde, kommt die Installation ohne manuelle Konfiguration und damit hohem Zeitaufwand aus. Von Nachteil ist allerdings, dass durch den Betrieb von zusätzlicher Hardware die aus dem Sensor einen vollwertigen Knoten machen, der Energieverbrauch meist um ein Vielfaches über dem liegt, was der Sensor alleine benötigt. \citep{voigt_illustration_2013} \citep{akyildiz_survey_2002} \citep{vieira_survey_2003} \citep{resch_standardisierte_2012}

Benjamin Voigt, siehe \citep{voigt_illustration_2013}, beschreibt in seinem Artikel unter anderem ein drahtloses Sensornetzwerk zur Überwachung eines Vulkans, dessen Zuverlässigkeit in Bezug auf die Erkennung von Ereignissen in einem zeitlich abgegrenztem Experiment untersucht wurde. Alle relevanten Ereignisse wurden zwar zuverlässig erkannt, darüber hinaus aber 99\%  nicht relevante Ereignisse. Zu den schlechten Ergebnissen führten vor Allem Fehler der Software und der Algorithmen. Das lässt den Autoren die Schlussfolgerung ziehen, dass solch ein System noch nicht für den operativen Dienst verwendbar ist. Hauptsächlich auch deshalb, weil die Lösungsansätze (höherer Betreuungsaufwand) bei Netzwerken mit einer realistischen Anzahl von Knoten (<1000) nicht finanzierbar wären.

Ein weiteres Beispiel für autonome Sensornetzwerke wäre ein Teil des in der Einleitung beschriebenen Projektes \gls{SOSEWIN}. Wie bereits beschrieben wird geplant die bestehenden Netzwerke durch ein Netzwerk aus autonom agierenden Sensoren gerade im städtischen Bereich zu erweitern. Dadurch dass die autonomen Sensoren ohne besonderes Fachwissen installiert werden können, sind auch Privatmenschen in der Lage Sensoren zu erwerben und zu betreiben, und damit zum Schutz der Gemeinschaft vor Katastrophen beizutragen.

\subsection{OGC Sensor Web Enablement}
Das \newacronym{SWE}{SWE}{Sensor Web Enablement} \gls{SWE} ist eine Initiative des \newacronym{OGC}{OGC}{Open Geospatial Consortium} \gls{OGC} und beinhaltet die Beschreibung verschiedener Standards zur Vernetzung von Sensoren im Web \citep{botts_ogc_2008}\citep{bermudez_ogc_2011}. Das \gls{OGC} ist die weltweit größte Organisation die sich um die Standardisierung des Datenverkehres von räumlichen Informationen bemüht. Das \gls{SWE} dient nicht nur zur Verknüpfung von Sensoren und zum Aufbau eines Netzwerkes, sondern es ermöglicht, Sensoren in das Web zu integrieren. Die Architektur des \gls{SWE} beinhaltet Lösungen um Sensoren im Web zu finden und zu beschreiben, um die gemessenen Daten zu lesen und um die Sensoren zu steuern \citep{botts_ogc_2008}\citep{broring_new_2011}. Das \gls{OGC} \gls{SWE} bietet mit seinen einzelnen Bausteinen des Sensor Web folgende Funktionalitäten:

\begin{description}
\item[Finden] Im Englischen steht der Ausdruck ``Discover'' für die gesamte Tätigkeit rund um das Suchen und Finden von Daten im Web. Diese Funktionalität soll das Finden der Messdaten und der Sensoren ermöglichen und vereinfachen.
\item[Sensor beschreiben] Diese Funktion liefert Metadaten zu den Sensoren wie etwa dessen generelle Eigenschaften, Arbeitsweise und Genauigkeit.
\item[Daten holen] Üblicherweise enthalten Webservices die mit Daten arbeiten eine Funktion die englischen ``Get Daten'' benannt ist. Mit dieser Funktion werden die angeboten Daten geholt.
\item[Aufgabenplanung] Um überhaupt Messergebnisse erhalten zu können muss die Arbeit des Sensoren geplant beziehungsweise werden angewiesen werden. Dies geschieht mit Hilfe dieser Funktion.
\item[Sensor abonnieren] Diese Funktion ermöglicht es passiv alle Ereignisse die der jeweilige Sensor zu vermelden hat zu erhalten.
\end{description}

\citep{botts_ogc_2008}\citep{broring_new_2011}

Die Standardisierung von Austauschprotokollen, Datentypen und Beschreibungssprachen ermöglicht es räumliche Daten einfacher und plattformübergreifend über das Web zu verbreitet und vereinfacht zu nutzen. Einzelne Standards des \gls{OGC} wurden bereits als \newacronym{ISO}{ISO}{International Standardisation Organisation} Standard eingetragen. Dadurch wird eine weltweit einheitliche Verfügbarkeit von Sensordaten gefördert. Auch andere Organisationen oder Initiativen die sich um Standardisierung von Geodaten bemühen orientieren sich an den Standards des \gls{OGC} und tragen damit zu einer Harmonisierung der Geoinformations Technologie bei. Das \gls{OGC} wiederum bemüht sich ebenfalls um ein Harmonisierung seiner eigenen Standards mit anderen wie zum Beispiel der IEEE 1451 ``Smart Transducer'' Standard Serie \citep{botts_ogc_2008}. Im Rahmen der \gls{SWE} Initiative wurden folgende Standards entwickelt:

\begin{description}
\item[Observation \& Measurement Schema] Der \newacronym{OM}{OM}{Observation and Measurement}\gls{OM} Standard beschreibt eine spezifische Vorgehensweise bei Messen von Objekteigenschaften. Diese Vorgehensweise kann -und soll- bei einer großen Bandbreite von Untersuchungen angewendet werden. Das Schema dient außerdem dazu Messergebnisse in vergleichbarer Form darzustellen. Eine Beobachtungen (englisch: Observations) wird im Rahmen dieses Schemas als Ereignis beschrieben, dessen Ergebnis mit einem numerischen Wert ein Phänomen beschreibt. Dabei werden Beobachtungen immer einem bestimmten Objekt des Interesses und der verwendeten Methode zugeordnet. Siehe auch \citep{cox_observations_2011}.
\item[Sensor Model Language] Die \newacronym{SensorML}{SensorML}{Sensor Model Language}\gls{SensorML} dient der einheitlichen Beschreibung von Sensoren und Prozessen. Diese \gls{SensorML} Dokumente sind notwendig um die Eigenschaften der Sensoren im Web eindeutig zu beschreiben und dienen dazu die Sensoren im Web besser zu finden. Sie enthalten unter Anderem Angaben über Position, Arbeitsweise und generelle Eigenschaften der Hardware. Die Arbeitsweise eines Sensoren wird detailliert mit Eingangswerten, Ausgangswerten, Parametern und Methoden beschrieben und dient dazu besser nachvollziehen zu können wie die Messerergebnisse zustande kommen. Ein Beispiel wäre das \gls{GPS} dessen Sensoren intern bereits die gemessenen Werte (Zeitangaben über Signal-Laufzeiten) prozessieren um sie überhaupt verwendbar zu machen. Siehe auch \citep{botts_opengis_2007}.
\item[Transducer Markup Language] Die \newacronym{TransducerML}{TransducerML}{Transducer Markup Language}\gls{TransducerML} dient als Protokoll dem Austausch von Informationen zwischen der Hardware und dem Sensor Web. Der deutsche Übersetzung ``Energiewandler'' für den englische Begriff ``Transducer'' steht für die eigentliche Sensor Hardware. In der \gls{TransducerML} wird der Begriff ``Transducer'' für das Superset aller vorhandenen Sensoren und Bedienungselemente verwendet. 
%%%   Hier nochmal umformulieren!!!!!!!!!!!!!!!!!!!!!!!
\gls{TransducerML} Dokumente enthalten Informationen über die Sensoren, dienen aber auch dazu Messwerte in vergleichbarer Form zu speichern, unabhängig von der sie produzierten Hardware. Zusammengefasst stellt die \gls{TransducerML} ein Modell für das Bereitstellen von Echtzeitdaten zur Verfügung. Die Daten sind mit einem Zeitstempel und einer Referenz auf das sie produzierende Sensorsystem eindeutig nachverfolgbar.
\item[Sensor Observations Service] Der \newacronym{SOS}{SOS}{Sensor Observations Service}\gls{SOS} steht als Web-Dienst beziehungsweise \gls{API} zur Verfügung um die Sensoren zu verwalten und die Daten abzurufen und zu filtern. Der Dienst ist der direkte Kommunikationsweg zwischen dem Nutzer und dem System. Das Ziel das das \gls{OGC} mit dem \gls{SOS} verfolgt ist, eine einheitliche Methode zur Verfügung zu stellen mit der die Messungen aller unterschiedlicher Sensoren abgerufen werden können. Siehe auch \citep{na_sensor_2007}.
\item[Sensor Planning Service] Der \newacronym{SPS}{SPS}{Sensor Planning Service}\gls{SPS} dient ebenfalls der Kommunikation zwischen dem Nutzer und dem System. Im Grunde kann der Nutzer über den \gls{SPS} dem System Beobachtungs-Aufträge erteilen. Dabei werden mehrere Aufträge beziehungsweise die Anfragen an verschiedene Sensoren gebündelt an den \gls{SPS} übermittelt. Über den \gls{SPS} können außer dem Anfragen von Daten an sich auch vergangene Anfragen geändert, zurückgenommen oder dessen Status abgefragt werden. Zudem bietet er die Möglichkeit Informationen über weitere \gls{OGC} Webservices, welche sich auf die Daten der Anfrage beziehen, zu erhalten.
\item[Sensor Alert Service] Mit dem \newacronym{SAS}{SAS}{Sensor Alert Service}\gls{SAS} können Nutzer über Ereignisse automatisch informiert werden. Der \gls{SAS} stellt ein Interface dar, über das Sensorknoten Beobachtungen anbieten können. Der Service ist damit nicht ein klassischer Mitteilungsservice, sondern eine Kartei (englisch: ``registry'') für zur Verfügung stehende Benachrichtigungs-Angebote. Möchte ein Nutzer über Ereignisse informiert werden, meldet er sich bei dem \gls{SAS} and. Dieser leitet die Anfrage den passenden Nachrichtenserver weiter, der dann die gewünschten Benachrichtigungen versendet.  Der \gls{SAS} ist bislang noch nicht als vollwertiger Standard, sondern nur als sogenanntes ``Best Practices Paper'' von der \gls{OGC} veröffentlicht.
\item[Web Notification Services] Für den asynchronen Austausch von Informationen zwischen verschiedenen Services und dem Nutzer dient der \newacronym{WNS}{WNS}{Web Notification Service}\gls{WNS}. Besonders bei Langzeittransaktionen ist eine asynchrone Kommunikation notwendig um den Nutzer oder einem anderem Service die Möglichkeit zu bieten jederzeit einzugreifen. Der \gls{WNS} ist ebenfalls noch nicht als vollwertiger Standard, sondern nur als ``Best Practices Paper'' von der \gls{OGC} veröffentlicht.
\end{description}

\citep{botts_ogc_2008}\citep{woolf_gigas_2008}\citep{kunkel_teodoor:_2012}\citep{walkowski_sensor_2008}

Das \gls{OGC} hat mit der \gls{SWE} Initiative eine Komplettlösung geschaffen, die es ermöglicht verschiedenartige Sensoren in eine sogenannte \newacronym{SDI}{SDI}{Spatial Data Infrastructure -deutsch: Geodaten Infrastruktur-} \gls{SDI} einzubinden. Damit ist auch eine Visualisierung von Sensordaten eingebettet in andere Webservices wie etwa einem Kartenservice möglich \citep{broring_new_2011}. Nähere Informationen zum Aufbau und Interaktion zwischen den einzelnen Bausteinen des Sensor Web werden in den nachfolgenden Kapiteln zu den Themen Geodateninfrastruktur und Softwaredesign gegeben.

\subsubsection{Sensor Observation Service}
\subsubsection{Sensor Alert Service}
\subsubsection{Sensor Planning Service}
\subsubsection{Sensor Instance Registry}
\subsubsection{Sensor Observable Registry}
\subsubsection{Sensor Modelling Language}
\subsubsection{Metadaten}

\subsubsection{Existierende Lösungen}
\paragraph{$52\,^{\circ}$North}
Das Framework $52\,^{\circ}$North stellt eine der umfangreichsten und frei verwendbaren Implementierungen für das \gls{SWE} dar und wird unter Anderem in dem Projekt \gls{GITEWS} (siehe Einleitung) verwendet.

\paragraph{istSOS}
Um für eine Risiko Management system in der Schweiz die Funktionalitäten der bisherigen \gls{SOS} Implementierungen zu erweitern entwickelte das schweizer Institut \newacronym{IST}{IST}{``Instituto scienze della Terra'' -deutsch: Institut für Geowissenschaften-}\gls{IST} 2010 eine eigene Software mit dem Namen \newacronym{istSOS}{istSOS}{Instituto scienze della Terra Sensor Observation Service}\gls{istSOS}. Die bereits existierenden Implementierungen, wie etwa das oben beschriebene $52\,^{\circ}$North  oder MapServer, erweitert das \gls{istSOS} System unter anderem um eine Unterstützung von unregelmäßigen Zeitreihen und enthält Mechanismen für die Qualitätskontrolle oder Messkorrekturen \citep{cannata_istsos_2013}. 

\gls{IST} entwickelte das System \gls{istSOS} komplett in Phython und vertreibt die Software unter den Lizenzbedingungen der \newacronym{GNU-GPL}{GNU-GPL}{GNU General Public License}. Das System basiert auf einem Apache Webserver \citep{apache_software_fundation_welcome!_2014} und stützt sich auf das räumliche Datenbanksystem PostgreSQL-PostGIS \citep{postgis_project_steering_committee_postgis_2014}. Neben dem \gls{SOS} enthält das \gls{istSOS} System den ``wa-Service'', der eine Reihe von Konfigurations- und Administrationswerkzeuge zur Verfügung stellt. Zusätzlich zu den Serverseitigen Implementierungen wurde auf Client-Seite auch die in JavaScript geschriebene Benutzeroberfläche ``wa-Interface'' geschaffen. Die Software-Architektur des Systems ist verschachtelt aufgebaut. Die Phyton Biliothek ``istsoslib'' stellt den Kern dar und ermöglicht Zugriff auf die Standard \gls{SOS} und auf \gls{istSOS} spezifische Funktionalitäten. Die Bibliothek ``walib'' ermöglicht als Zwischenschicht den Zugriff auf den wa-Service. Weitere Details zu der Software-Architektur finden sich in \citep{cannata_istsos_2013}\citep{cannata_istsos:_2010}.

\begin{figure}[H]
	\centering
 	 \includegraphics[scale=1]{graphics/istSOS_consumer-producer.png} 
	\caption{Interaktion mit dem istSOS System dargestellt mit einem UML Sequenzdiagramm jeweils für einen Datenkonsumenten und einen Datenproduzenten \citep{cannata_welcome_2014}}
	 \label{fig:istsos_consumer-producer}
\end{figure}

Wie bei den meisten \gls{OGC} Services wird auch mit dem \gls{istSOS} über den Austausch von standard Nachrichten über das \gls{HTTP} Protokoll kommuniziert. Die  \ref{fig:istsos_consumer-producer} zeigt sowohl für den Datenkonsumenten als auch für den Datenproduzenten Anfragen an den Service die mindestens in einer \gls{SOS} Implementierung enthalten sein müssen. Die Nachrichten werden beispielsweise mittels ``HTTP POST'' an den Service übermittelt, und dieser sendet die Antwort in Form von \newacronym{XML}{XML}{Extensible Markup Language}\gls{XML}. 

\section{Dateninfrastruktur}
\subsection{Datentypen und Datenformate}
\subsubsection{Keyhole Markup Language}
\subsubsection{Geographic Markup Language}
\subsection{Geodatenbanken}
\subsubsection{PostGIS}
\subsubsection{Oracle Spatial}
\subsection{Geodateninfrastruktur}
\subsubsection{Web Map Service}
\subsubsection{Web Feature Service}
\subsubsection{Web Catalogue Service}

\section{Analysemethoden}
\subsection{Methode der finiten Elemente}
\subsection{Datenkonsistenz}

\section{Mobile Dienste}
In der Einleitung bin ich bereits umfangreich auf die besonderen Charakteristiken von mobilen Diensten eingegangen, insbesondere auf die nicht-funktionellen Eigenschaften. In diesem Kapitel skizziere ich die aktuellen Forschungen in diesem Bereich und gehe detailliert darauf ein welche Rolle sie für diese Arbeit spielen können.

Der bereits zitierte Martin Breunig, siehe \citep{breunig_entwicklung_2003}, wies bereits früh auf das Potential von Geographischen Informationen für mobile Dienste hin. Die wesentlichen Teile eines solchen Systems haben sich bis heute kaum verändert, deren heutiger Entwicklungsstand ist jedoch seit dem um einiges weiter fortgeschritten. Breunig beschreibt ebenfalls die Nutzung von standardisierten Schnittstellen (siehe Kapitel Dateninfrastruktur), die besonderen Anforderungen der System-Architektur sowie die Datenvisualisierung als Kernpunkte eines Konzeptes für ein mobiles System.


\chapter{Softwareentwurf}
