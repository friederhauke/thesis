\chapter{Einführung in die Problematik}
Diese Masterarbeit beschreibt den potentiellen Nutzen eines grafischen Management Systems für die Gebäudeüberwachung mittels verschiedener stationärer und mobiler Sensoren. Anhand eines Beispieles soll geklärt werden welche Technologien eingesetzt werden müssten um diese Aufgabe zu lösen, und worin genau der Mehrwert eines solchen Systems gegenüber konventioneller Vermessungssoftware liegt. Die vorgeschlagenen Technologien werden zum Abschluss der Arbeit teilweise praktisch angewandt respektive umgesetzt. Dieser Prototyp soll exemplarisch demonstrieren wie solch ein System arbeiten wird. Kernfragen der Masterarbeit ist, wie kann dieses System Feldingenieuren, die im Bereich der Gebäudeüberwachung arbeiten, generell in den folgenden Bereichen unterstützen:
\begin{itemize}
\item Planung und Optimierung der allgemeinen Arbeiten im Feld
\item Datenkommunikation mit dem Büro
\item Vorabauswertung der Messwerte direkt im Feld
\end{itemize}
Die Klassische Arbeit von Vermessungsingenieuren besteht aus dem praktischen Teil der im Feld durchgeführt wird, und den anschließenden Auswertungen und der Interpretation im Büro. Durch die räumliche Trennung dieser Beiden dennoch sehr ineinander verzahnten Aufgaben entsteht oftmals eine Verzögerung in den Abläufen und erhöhtes Risiko für vermeidbare Fehler in den Arbeitsabläufen. Das System soll so konzeptioniert sein, dass es die Lücke schließt, zum Einen um die Effizienz der Arbeiten zu erhöhen, und zum Anderen um Fehler bei den Messungen oder der Datenmigration zu erkennen oder von vorn herein zu vermeiden. 

Heutzutage ist das Bearbeiten von unterschiedlichen Arbeiten an einem Ort zur gleichen Zeit keine Vision mehr. Mobile Endgeräte wie \"Smartphones\" oder \"Tablet-Computer\" vereinfachen Arbeitsabläufe, und helfen Zeit zu sparen. Bei den Arbeiten im Feld sind mobile Endgeräte bereits ständig präsent, dennoch helfen sie lediglich bei wenigen Aufgaben wie der papierlosen Bürokratie, der Email-Kommunikation mit dem Büro oder den betrachten vorheriger Messerergebnisse. Vergleiche auch \citep{breunig_entwicklung_2003} \citep{breunig_vision_2003} Das hier konzipierte System hingegen weißt folgende drei Hauptvorteile gegenüber der aktuellen Nutzung von mobilen Endgeräte auf:
\begin{itemize}
\item Kernfunktionalität ist das Assistieren des Vermessungsingenieurs. Das heißt es soll Hilfestellung beim Verstehen der Messungen geben (zum Beispiel durch den Vergleich der aktuellen Messungen mit vergangenen Messreihen). $\rightarrow$ Das Teilen von Informationen zwischen Feld und Büro führt zu einem besseren Kenntnisstand während der Arbeiten im Feld.
\item Fehlmessungen wie sie etwa beim vertauschen von Positionen geschehen sollen vermieden werden indem die Messergebnisse direkt nach Eingabe in das System auf ihre Konsistenz hin überprüft werden.
\item Für die Aufnahmen im Feld brauchen vorherige Messreihen nicht umständlich exportiert zu werden, sondern diese können von dem Mobilen System direkt von dem gemeinsamen Daten-Server abgerufen werden.
\end{itemize}
Das hier konzeptionierte System ist keine Alternative für klassische Vermessungssoftware, ist es eine Ergänzung und eine Brücke von den Sensoren direkt zu dem System. Es ermöglicht die Kombination von klassischer Ingenieurvermessung mit Sensornetzwerken.


\section{Vergleichbare Systeme}
Das Konzept der vernetzten Sensoren wurde bereits in einigen Systemen erfolgreich angewandt. Gerade bei Problemen die sich auf einen größeren Raum und über eine längere Zeit erstrecken ist es häufig nicht mehr möglich mit Einzelmessungen genug Informationen für eine Analyse zu erhalten. Ein wichtiges Argument für den Einsatz solcher Netzwerke ist natürlich auch die Möglichkeit alle verbundenen Sensoren bequem und ohne den Einsatz zusätzlicher Mittel von einem Ort aus kontrollieren zu können. Fehler in der Software beeinflussen damit natürlich auch das gesamte System, die Fehlersuche beschränkt sich dann aber auch nur auf einen Punkt. Nicht zuletzt spielt auch die Ökonomische Betrachtung solcher Systeme eine wichtige Rolle, Zentralisierung von Kräften bedeutet auch eine REduzierung von Kosten. Wenige Spezialisten ersetzen eine große Anzahl an Generalisten im Feld. Eine tiefer gehende Diskussion von Sensornetzwerken im allgemeinen und des Prototypen im Speziellen ist im dritten Teil der Arbeit zu finden.

\begin{wrapfigure}{r}{0.5\textwidth}
  \begin{center}
 	 \includegraphics[scale=0.45]{graphics/GITEWS_Warning_Centre_01.jpg} 
	\caption{Tsunami Warnzentrum in Indonesien, GITEWS 2011}
  \end{center}
\end{wrapfigure}
Ein sehr prominentes Beispiel für den Einsatz von Sensornetzwerken ist das Deutsch-Indonesische Tsunami Frühwarnsystem \newacronym{GITEWS}{GITEWS}{German Indonesian Tsunami Early Warning System} \gls{GITEWS} Nach dem verheerenden Tsunami von 2006 im Pazifischen Ozean, der vor allem in Indonesian viele Opfer forderte, beschloss Deutschland den Aufbau eines Frühwarnsystems, dass die Menschen in Indonesien besser vor Tsunamis warnen sollte. 
Tsunamis werden meist durch eine spezielle Art von Erdbeben ausgelöst, bei der sich der Seeboden in vertikaler Richtung bewegt und damit Wasser verdrängt. Dadurch entsteht ein Berg aus Wasser, der dann in alle Richtungen ausläuft. Somit können Tsunamis vorhergesagt werden, indem alle vorkommenden Seebeben daraufhin untersucht werden ob und wo ein Tsunami entstehen könnte. Das \gls{GITEWS} System basiert auf einem dichten Netz aus Seismographen, \newacronym{GPS}{GPS}{Global Positioning System} GPS-Stationen und Pegelstationen rund um Indonesien. Seismographen an sich können zwar Erdbeben registrieren, sind aber alleine nicht in der Lage Aussagen über Epizentrum, Art und Stärke des Erdbebens zuzulassen. Die Vernetzung der Stationen stellt demnach eine der essentiellen Eigenschaften eines Erdbebenwarnsystems dar. Und damit die Warnung früh genug verbreitet werden kann müssen die Daten innerhalb eines äußerst kurzen Zeitraums erfasst und ausgewertet werden. Am besten gelingt das indem die Daten in Echtzeit übermittelt werden, und kontinuierlich ausgewertet werden. Das System dient als Kommunikations-Knoten zwischen allen Sensoren, als Auswertungssystem für die Daten und auch als Visualisierungsplattform für die Ergebnisse. \citep{lauterjung_gitewstsunami-fruhwarnsystem_2011} \citep{strobl_geodatenmanagement_2007} \citep{spahn_experience_2010}

In dem Projekt \newacronym{SOSEWIN}{SOSEWIN}{Self-Organising Seismic Early Warning Information Network}\gls{SOSEWIN} beschäftigen sich Wissenschaftler ebenfalls mit einem Netzwerk aus Seismometern und weiteren Messinstrumenten. Bei diesem Projekt sollen jedoch nicht vor Tsunamis, sondern vor Erdbeben im Raum Istanbul gewarnt werden. 

\begin{wrapfigure}{r}{0.5\textwidth}
  \begin{center}
 	 \includegraphics[scale=0.33]{graphics/SOSEWIN_Sensor_Bridge.jpg} 
	\caption{Sensor an der Sultan Mehmet-Brücke über den Bosporus, SOSEWIN 2011}
  \end{center}
\end{wrapfigure}
Dazu werden engmaschig Seismometerstationen und zudem im Stadtgebiet Bewegungssensoren  und GPS-Empfänger installiert. Aus der Kombination der verschiedenen Messwerte können einige Sekunden vor einem Erdbeben Warnungen ausgegeben werden. Dieses Sensornetzwerk stellt ein sich selbst organisierendes und über TCP/IP kommunizierendes Netzwerk dar. Anders als bei anderen Netzwerken organisiert nicht ein zentraler Knoten alle Sensoren, sondern die Sensoren sind jeweils mit einem Mini-Computer verbunden, der dann aktiv mit den anderen Knoten kommuniziert. Geplant ist auch eine Erweiterung des bestehenden Netzwerkes indem Privathaushalte Sensorkomponenten erwerben, und sich dadurch aktiv an der Frühwarnung beteiligen. Diese Möglichkeit besteht nur weil die Sensoren "quasi-autonom" funktionieren. Zusätzlich sollen auch kritische Infrastrukturen überwacht werden. Testweise wurden bereits Instrumente an der Sultan Mehmet-Brücke über den Bosporus installiert um deren charakteristischen Eigenschwingungen zu messen. Gerade diese Brückenüberwachung hat gewissen Ähnlichkeit mit dem hier konzipierten Beispielnetzwerk, das zu Grunde liegende System allerdings unterscheidet sich maßgeblich. \citep{luhr_sekunden_2011}