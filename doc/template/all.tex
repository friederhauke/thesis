
\documentclass[a4paper,11pt]{article}

\usepackage{amsmath,amssymb,amsfonts,amsthm}    % Typical maths resource packages
\usepackage{graphicx}                           % Packages to allow inclusion of graphics
\usepackage{hyperref}                           % For creating hyperlinks in cross references
\usepackage[authoryear]{natbib}                 % literature reference style
\usepackage{url}
\usepackage{alltt}
\usepackage{listings}
\usepackage{rotating}
\usepackage{cite}

\usepackage{multicol}

\usepackage{multirow}
\usepackage{graphicx}
\usepackage{subfig}
\usepackage{geometry}                % See geometry.pdf to learn the layout options. There are lots.
\geometry{a4paper}                   % ... or a4paper or a5paper or ... 
%\geometry{landscape}                % Activate for for rotated page geometry
\usepackage[parfill]{parskip}    % Activate to begin paragraphs with an empty line rather than an indent
\usepackage{graphicx}
\usepackage{subfig}
\usepackage{amssymb}
\usepackage{amsmath}
\usepackage{epstopdf}
\usepackage{multicol}
\usepackage{amsmath}
\usepackage{listings}
\usepackage{float}

\usepackage{multirow}
\usepackage{color}


% -------------------------------
\usepackage{natbib}% --- some layout definitions ---
% -------------------------------

% define topline
\usepackage[automark]{scrpage2}
\pagestyle{scrheadings}
\automark{section}
\clearscrheadings
\ohead{\headmark}
\cfoot{\pagemark}

% define citation style


% define page size, margin size
\setlength{\headheight}{1.1\baselineskip}
\voffset=-3cm
\hoffset=-3cm
\textheight24cm
\textwidth16.5cm
\topmargin1cm
\oddsidemargin3cm
\evensidemargin3cm

% define line line spacing = 1.5
\renewcommand{\baselinestretch}{1.5}

% define second level for `itemizing'
\renewcommand{\labelitemii}{-}

% --------------------------------------
% --------------------------------------
% --------------------------------------
% --- the structure the tex document ---
% ---  (this our recommendation) -------
% frontmatter:
%   - titlepage (mandatory),
%   - acknowledgement,
%   - abstract,
%   - table of contents (mandatory),
%   - list of abbreviations (not mandatory),
%   - list of figures (not mandatory),
%   - list of tables  (not mandatory) .
%
% body of the thesis (the structure of the thesis body is not mandatory, but the list of literature is mandatory):
%   - introduction,
%   - methods,
%   - data,
%   - results,
%   - conclusion,
%   - literature (mandatory),
%   - appendix (figures, tables).
%
% last page:
%   - declaration of authorship (mandatory).
% --------------------------------------
% --------------------------------------
% --------------------------------------

\begin{document}

% -------------------------------
% --- frontmatter: Title page ---
% -------------------------------
\thispagestyle{empty}

\begin{center}
\vspace*{\fill}

    {\Large{\bf Concept of an mobile assisting system for multi-sensor buildings-observation}} \vspace{0.5cm}
    
    {\Large{Entwurf eines mobiles Assistenzssystem für multisensorische Bauwerksüberwachung}} \vspace{0.5cm}


    {\normalsize EXPOSE: Master Thesis Euteneuer}\\\vspace{0.5cm}
    {\normalsize Technische Universität Berlin \\
    Institute für Geodäsie und Geoinformationstechnik\\
	Fachgebiet Methodik der Geoinformationstechnik\\
	superviesd by:\\	
	Prof. Frank Neitzel\\
	Thomas Becker}\vspace{1cm}

    {\normalsize by \\\vspace{0.5cm}
    {\bf Frieder H. Euteneuer}} \vspace{1cm}
		

    {\normalsize Berlin, \today}
\vfill
\end{center}




% ------------------------------------
% --- frontmatter: Acknowledgement ---
% ------------------------------------
\newpage
\pagenumbering{roman}   % define page number in roman style
\setcounter{page}{1}    % start page numbering
%\input{acknowledgement}

% -----------------------------
% --- frontmatter: Abstract ---
% -----------------------------
%\newpage
%\input{abstract}



% -----------------------------
% --- frontmatter: Contents ---
% -----------------------------
\newpage
\tableofcontents
%\clearpage


% ----------------------------------------------------
% --- frontmatter: List of Figures (not mandatory) ---
% ----------------------------------------------------
\newpage
\addcontentsline{toc}{section}{List of Figures}
\ohead[]{\rightmark}
\listoffigures


% ---------------------------------------------------
% --- frontmatter: List of Tables (not mandatory) ---
% ---------------------------------------------------
\newpage
\addcontentsline{toc}{section}{List of Tables}
%\listoftables



% -------------------------------
% --- main body of the thesis ---
% -------------------------------
\newpage
\setcounter{page}{1}    % start page numbering anew
\pagenumbering{arabic}  % page numbers in arabic style
\setcounter{secnumdepth}{4}


% 	PASTE HERE YOUR RESULT DOCUMENTS SEPARATED WITH NEWPAGES AS YOU CAN SEE BELOW
% 	AN ORDERING BY THE TWO PHASES AND BY THER THREE GROUPS WOULD BE USEFUL
%	--> PHASE 1; GROUP1; GROUP2; GROUP3; PHASE2; GROUP1 ...
%
%EXAMPLE
\part{Data Acquisition, Data Integration and Creation of LOD3 Model}
\newpage

\chapter{ %group 1
\section{Group 1 - Extraction of energy relevant attributes from 3D city model}
}
\subsection{Introduction}
The task for group 1 in phase 1 was to specify energy relevant attributes and try to calculate or extract them from the semantic city model.\\
First of all a connection to the database was necessary to establish, and the following three subtasks where reasonable to define.
With the specification of the energy relevant attributes inside of the database, a data basis for the different algorithms was created. This algorithms can be differentiated into data analysis algorithms basing on the pure values stored in the database and geometry analysis dealing with the geometry information in the database.\\
Our work was restricted to the area of Berlin Moabit, a small subset of the full Berlin model. For the Java algorithms the database information where exported as .gml files and the obtained output was saved as .csv files, which were able to be re-uploaded to the database. All data kept their necessary identifier.\\
As important values for an energy atlas following parameter where determined:
\begin{itemize}
\item Airvolume of building to be able to calculate the Surface-to-Volume-Ratio
\item Number of storeys which can be calculated by using the average storey-height of the specific building
\item Orientation of the wallsurfaces of each building to get information about neighbourhood relations between buildings and inner- and outer walls
\end{itemize}

\subsection{Data Analysis}
Each building stored in the database contains at least one groundsurface, one roofsurface and three wallsurfaces. Based on these information, the volume of the buildings, the orientation of the walls (whether it is a outer or neighbouring wall), therewith also the surface-to-volume-ratio and the outer-wall orientation for sunlight-heating-effects can be calculated. Also necessary for the average storey height and important for the calculations is the building’s function as it is residential, public or industry.\\
Moreover, we identified some missing energy relevant data like the building's year of construction. With this it is possible to estimate the building's material, the mean size of windows, the type of windows, and the mean story height. In addition, the topological relations between buildings are missing, as well as the number of people living in the building resp. number of flats, and the behaviour classes of the people living in the building (to estimate their energy consumption).

\subsection{Geometry Analysis}
In the geometry analysis we mainly focused on the surface area to volume ratio as well as the outer orientation of the walls. Carrion \citep{carrion2010} defines the surface area to volume ratio as follows:
\begin{quote}
"The S/V is the ratio of the aggregated area of all surfaces which transmit energy to the surrounding (wall surfaces touching other buildings are not considered) and the volume of the building."
\end{quote}
To calculate the S/V ratio, we created two subtasks. One task is to calculate for each building the area which is touching neighbouring buildings as well as to calculate the sum of all surface areas of one building. The second task is to calculate the volume of each building.

\subsection{Calculation of wall-surface intersection}
Important to know is, that the S/V does not include wall surfaces touching other buildings, and therewith those wallsurfaces must be detected using the information which are stored in the database. For this issue the following workflow to calculate wall surfaces which are not touching other buildings was developed:
\begin{enumerate}
\item Calculate neighbouring buildings
\item Calculate area touching other buildings (for each building)
\item Calculate sum of surface areas (ground-, roof-, wallsurfaces)
\item Subtract intersecting area with neighbouring building from sum of surface areas
\end{enumerate}

\subsubsection{Calculate neighbouring buildings}
The calculation of neighbouring buildings is done, to minimize the candidate set for the calculation of the intersection of two buildings. Two buildings are neighboured if the wall surfaces of the two buildings are within a certain distance, and the angle between wall surfaces is smaller than a defined threshold (see figure \ref{fig:neighbours3d}). 
Since we did the calculation in Java with the JTS Topology Suite (JTS), the problem was that JTS is not able to do calculations on 3D geometries. A workaround is to make the assumption, that the wall surfaces are vertical and parallel to each other. Then a projection of the walls into 2D can be done by ignoring the Z coordinates.
\begin{figure}[h]
	\centering
 	 \includegraphics[scale=0.35]{phase1/group1/neighbours3d.png}
	\caption{Neighboured buildings in 3D.}
	 \label{fig:neighbours3d}
\end{figure}
This means, every wall surface becomes a linestring (see figure \ref{fig:neighbours2d}).
\begin{figure}[h]
	\centering
 	 \includegraphics[scale=0.8]{phase1/group1/neighbours2d.png}
	\caption{Neighboured buildings in 2D.}
	\label{fig:neighbours2d}
\end{figure}
Then we can test if two walls represented by linestrings are within a certain distance and the angle between the linestrings is smaller than a threshold using JTS.

\subsubsection{Calculating area touching other buildings}
After determining if two buildings are neighboured, it is possible to calculate the intersection area of touching wall surfaces. Since JTS is not able to do calculations on 3D geometries, as mentioned above, the walls need to be transformed to 2D again. Therefore the coordinate system has to be transformed so that the wall surfaces are lying in the YZ-plane. Then, the X coordinate can be ignored and the intersection of the two walls can be calculated.\\
To rotate the coordinate system so that the wall surfaces are lying in the YZ-plane, the normal vector of one of the two walls has to be calculated. With this the rotation angle $\alpha$ can be calculated as the angle between the normal vector on the vector [1 0 0], which is the normal of the YZ-plane. Then both wall surfaces are rotated with this rotation angle $\alpha$. Figure \ref{fig:rotation_matrix} shows the used rotation matrix.
\begin{figure}[h]
	\centering
 	 \includegraphics[scale=0.5]{phase1/group1/rotation_matrix.png}
	\caption{Rotation matrix used for calculating intersection.}
	\label{fig:rotation_matrix}
\end{figure}
If these rotated surfaces are intersecting, the real world surfaces are intersecting, too. Thus, the intersection of the rotated surfaces can be calculated as in the following listing:\\
\begin{lstlisting}[language=Java]
Geometry intersection=polygon.intersection(polygonNeighbour);
if (intersection instanceof com.vividsolutions.jts.geom.Polygon) {
  com.vividsolutions.jts.geom.Polygon intersectionPolygon 
  = (com.vividsolutions.jts.geom.Polygon) intersection;
  // unit is m^2
  intersectingArea = intersectionPolygon.getArea();
}
\end{lstlisting}

\subsubsection{Overall- and sharing wall area}
This is followed by the area calculation of the wall-, roof- and groundsurfaces for each building. These surface areas are summed up to get the overall surface area of each building.
These results (the surface areas of each building (building id)) are written to a .csv file to be able to import it back into the 3D city model database. Table \ref{table:result shared wall surfaces} shows some sample results.
\begin{table}[b]
\centering
\begin{tabular}{c  c  c}
building\_id & surface\_area & shared\_wall\_area\\
\hline						
BLDG\_0003000000432cd8 & 234.99638499102912 & 45.46462674214126\\
BLDG\_0003000000432c80 & 1265.0716547126067 & 529.4226535306225\\
BLDG\_0003000f000858d0 & 2201.2130139165056 & 975.798086083496\\
...\\
\end{tabular}
\caption{Subset of result .csv file containing shared wall surfaces.} 
\label{table:result shared wall surfaces}
\end{table}

\subsection{Geometry Analysis - Building Volume}
The building's volume can be calculated using several different approaches. The task is to have a volume which is a good approximation of the real existing air volume inside of the building, because it will be used afterwards for some energy-flow calculations. Necessary also for an operatively used approach is the automatisation of the algorithm and the potential for for its embedding into the other code.\\
The following approaches have been tested and checked whether they fulfil the requirements:
\begin{enumerate}
\item SQL algorithm/query
\item FME Software
\item ArcMap Software - 3D Analyst
\item ArcMap Software - Buildings to DEM
\end{enumerate}


\subsubsection{Volume calculation 1. approach: SQL}
Using SQL (Structured Query Language) as a possibility to directly access the database and analyse the stored information is the first upcoming option. For the calculation several Oracle spatial functions have been used to get the building area. Multiplying this with the building's height which is already stored in the database leads to a good estimation of the building's volume.\\
But the differences between the building's roof structure leads to wrong estimations of a large percentage of the building's volume, because only one height can be taken from the database which is the distance between the ground up to the highest roof point. The figure \ref{fig:build_h} is showing a good example of a possible wrong estimation: The different building parts are stored using only one building identifier, and therewith only the height of the highest part is stored inside of the database.
\begin{figure}[h]
	\centering
 	 \includegraphics[scale=0.3]{phase1/group1/build_h.png} 
	\caption{Example of a more complex roof sturcture.}
	\label{fig:build_h}
\end{figure}

\subsubsection{Volume calculation 2. approach: FME}
The second approach was using the software FME (Feature Manipulating Engine). It’s FME Data Inspector shows the stored geometry and the buildings do not contain ground-, wall- and roofsurfaces (see figure \ref{fig:fme}). But the surfaces were represented as not connected so a building is not containing one closed geometry.\\
Since the FME workbench is very complex also a lot of errors occur, and therewith this approach is not really stable.
\begin{figure}[h]
	\centering
 	 \includegraphics[scale=0.3]{phase1/group1/fme.png} 
	\caption{Visualisation of FME Data Inspector.}
	\label{fig:fme}
\end{figure}

\subsubsection{Volume calculation 3. approach: Arcmap}
The Software Arc Map offers an extension called 3D Analysed which can be used as an analysis tool for 3D objects.\\
In here it was possible to enclose the geometries of the buildings, but this procedure uses a shrinking of the building until every surface is completely touching its neighbours. Therewith the obtained volume of the building in systematically falsificated.

\subsubsection{Volume calculation 4. approach: Arcmap again}
For the last approach another functionality of Arc Map can be used: The creation of a rasterlayer for ground- and roof- geometries with the extension: "Add Buildings to DEM".\\
A complex sequence of operations have to be applied to first create a difference raster which is showing ground level and roof level and secondly calculating the volume using this raster.\\
For the automatisation the Arc Map Modelbuilder can be used. With this every step can be defined as an output and input of others. Figure \ref{fig:arcmap_raster} is showing the process.
\begin{figure}[h]
	\centering
 	 \includegraphics[scale=0.1]{phase1/group1/raster_ground.jpg} 
	 \includegraphics[scale=0.1]{phase1/group1/raster_roof.jpg}
	 \includegraphics[scale=0.1]{phase1/group1/raster_diff.jpg} 
	 \includegraphics[scale=0.1]{phase1/group1/arcscene_tin.jpg} 
	 \includegraphics[scale=0.1]{phase1/group1/modelbuilder.jpg} 
	\caption{Visualisation of the 4th approach to calculate the volume.}
	\label{fig:arcmap_raster}
\end{figure}

\subsubsection{Volume calculation - Comparison}
Since our knowledge about the buildings is only depending on our data, a valid number of volume cannot be calculated without any statistical analysis of the buildings and the different calculations.\\
The table \ref{table:comparison_volume} shows a comparison of the different obtained results which.
\begin{table}[b]
\centering
\begin{tabular}{c  c  c}
Approach & calculated volume\\
\hline						
SQL & $10227m^3$\\
FME & no result\\
3D Analyst & $6000m^3$\\
Arc Map & $7400m^3$\\
\end{tabular}
\caption{Comparison of the different volume calculation procedures.} 
\label{table:comparison_volume}
\end{table}

\newpage
\chapter{
\section{Group 2 - Creation of 3D Model}}
\subsection{Introduction}
The modelation of Cities as 3D City model got more famous in the last years. Different appraoches from different domains to model the real world are common. The most famous domains, which deal with the creation of 3D models are Computer graphics, Architectural aided design and geoinformation science. In computer graphic and architecturel models only the geometry is represented. 3D models in geoinformation science store next to the geometry also semantics. This enables a user to use the 3D models for further investigations. This project deals with the enrichment of such a city model with an additional semantic layer, which contains energy related attributes. For this, the data of the 3D City DB of Berlin Moabit is used. Because the buildings are only available in LoD2, although LoD3 is required for a reliable energy demand estimation first LoD3 buildings are modeled. To create LoD 3 buildings different approach to model 3D building models with from building photographs are adopted. SketchUP is used for the 
modelation process. These models are enriched with the mentioned energy related attributes. Finally the new LoD 3 models are integrated in the 3D City DB.

\subsection{Workflow}
To model LoD 3 buildings and enrich them with energy realted attributes, the work can be splitted in three main parts, data acquisistion, 3D modeling and data integration. Figure \ref{fig:workflowph1} depicts the content of each project phase.

\begin{figure}[ht]
	\centering
	\includegraphics[width=0.8\textwidth]{phase1/group2/figures/workflow_phase1.png}
	\caption{Work packages for phase 1 of the GIS project}
	\label{fig:workflowph1}
\end{figure}

To model a LoD 3 building additional information about the geometry of e.g. windows, doors, as well as high resolution textures are necessary. For the integration of energy related attributes additional semantic data like the number of flats or the heating system have to be aquired. In order to collect this data a questionnaire was developed in the first step. \\
These questionnaire was used to acquire the necssary data for individual buildings by an onsite inspection. \\
The collected data is used to create LoD 3 models with SketchUP in phase 2. 
\\
Finally the models are enriched with the energy related attributes that are needed for the objective of this project.
Furthermore the new 3D models are exported as CityGML files and integrated into the 3D City Database. \\
In addition it was tried to export the building models as Energy Plus files, in order to run a Energy simulation with different input parameters. Because several problems with the SketchUp plugin Open Studio, which provides functions to generate Energy Plus consistent building models, were encountered it was not possible to export a Energy Plus file. \\
Therefore the report concentrates on the development of LoD 3 building models. 
\\
From the work packages the project schedule was created as guide to manage the project within a specific time frame. The project schedule can be found in annex1

\subsection{Data acquisistion}
\subsubsection{Development of a questionnaire}
The questionnare should cover all relevant questions to collect the data, which is needed for the creation of LoD 3 buildings models and the enrichment of these with energy related attributes. Therefore the requirments for LoD 3 building models have to be known. A LoD 3 building geometry comprises surface as well as windows, doors and balconies. Furthermore high resolution textures have to be available. To decide which energy related attributes are neccessary the "Kurzverfahren Energie Profil" of the Institut fuer Wohnen und Umwelt (IWU 2003) was used as reference. The questionnare can be seperated into eight parts:
\begin{itemize}
\item Address
\item General building attributes: Year of Construction, Number of flats, functionality, etc. (Figure \ref{fig:questionnaire}
\item Ground plan: sketch to get a better orientation
\item Adjacent buildings: neighboorhood sitiuation of the building (free standing, neighboor)
\item Basement
\item Roof: type, material, etc.
\item Window: Number of windows, U-values
\item Walls: Type, Insolation, Thickness, etc.
\item Doors: Material
\item Heating system
\item Additional notes
\end{itemize}

To simplify the field work for most questions predefined choices were created. For example seven roof types are proposed. The whole questionaire can be found in annex 2.

\begin{figure}[ht]
	\centering
	\includegraphics[width=0.8\textwidth]{phase1/group2/figures/questionnaire.png}
	\caption{Developed questionnaire}
	\label{fig:questionnaire}
\end{figure}


\subsubsection{Onsite inspection and data acquisistion}

The area of interest of the project is Berlin Moabit. In this area six buildings were selected based on different criteria like functionality, neighboring situation and year of construction. Therefore one free standing young office building, one old (1961) residential building with two adjacent buildings, as well as three residential buildings with one adjacent building were selected. \\
For this six buildings the questionnare was answered and several images were taken. Additionaly measuremnts were performed to determine the thickness of the walls and at least one reference height in order to scale the building models correctly. 
The images are later on used to model the buildings. Therefore images of each corner and each surface have to be taken as shown in figure \ref{fig:images}.

\begin{figure}[ht]
	\centering
	\includegraphics[width=0.5\textwidth]{phase1/group2/figures/images.png}
	\caption{Constrains on image acquisition}
	\label{fig:images}
\end{figure}

Because most of the buildings have adjacent buildings it was not always possible to take proper images of the building corners. Furthermore trees and cars were in the field of view.\\
The questionnaire is very detailed and covers all necessary data. In reality it has to be admitted that it was not possible to gather all information proposed by the questionnaire. For example it was only possible to collect information about the heating system for one building. The determination of year of construction of the buildings was not accuratly, too. The gathered information about the roof, e.g. type and material, the windows and doors, e.g. glazing, and walls is reliable.
\\
\\
The collected data is reorganized for furhter processings in a new Data sheet. It is attached as annex 3.

\subsection{3D Modeling}
The creation of LoD 3 building models is the major part of this project phase. To model the building SketchUp, a 3D modeling software, is used. In addition several SketchUP plugins such as Geores, a plugin to export the SketchUP models to CityGML, or OpenStudio, a plugin to generate Energy plus convienent models, are used. Furthermore FME is used to attach attributes to the CityGML models. The 3D City DB Importer and Exporter tool, which was developed by the geodesy and Geoinformation department of the Technical University of Berlin, is used to validate the created CityGML files and to integrate them in the 3D City DB.

\subsubsection{Creation of building model}
The 3D building modeling is based on the acquired images from different perspectives.To model consistent LoD3 buildings a work pattern was development, which was followed for all buildings. In the following the single steps are pointed out.
 
\begin{itemize}
\item Definition of left and right vanishing points.
\item Photos are calibrated based on the assumption that there are 90° angles.
\item Set the origin to a corner of the building. 
\item Set the scale. This is done by drawing a line representing the measured reference distance and scaling it to the real world length. Reference distances are measurments of the door, window or the 2D foot print which is extracted from the existing LoD 2 model in the 3D City DB.
\item Digitize the shape of the building from the current image as far as possible.Continue with the next images in the same way.
\item Windows and doors are digitized by creating openings in the. 
\end{itemize}

The roof of the building cannot be created from the acquired images. The shapes of the roofs are estimated from the satellite images of google earth and modeleld without reference.

\begin{figure}[ht]
	\centering
	\includegraphics[width=0.5\textwidth]{phase1/group2/figures/Building_wiclef.png}
	\caption{Work packages for phase 1 of the GIS project}
	\label{fig:Building_wiclef}
\end{figure}

After the creation of the 3D model of a building the used images are projected on the surface of the models and used as textures. Because some images are highly distorted the textures don't correspond perfectly to the geometry. The following steps were followed to project the texture on the model.

\begin{itemize}
\item choose a picture
\item Mark the surface to which the texture should be applied 
\item Press the buttion "project texture from photos" from the "Match-Photo"-Dialog
\end{itemize}

Four of the six buildings were modeled with this workflow.


\subsubsection{Handling Layers in SketchUP}

For a successful export to cityGML the features of a building have to organized in a sophisticated layer structure.
Different layers are created for the model in order to identify each surface and opening as a unique entity in the exported cityGML file. The layer is created with sketch up following some standard rules to make each layer unique with an id. The workflow used for creating the layers is as follows:


\begin{itemize}
\item Highlighted a particular surface or part of the building model 
\item Click on ``create layer'' from the window button in sketch up
\item Assign a layer name according to following rules
\end{itemize}

This is repeated for all the surfaces and all the building part of the model. Features of each surface could be grouped or ungrouped for a single surface. For the purpose of this project the building features are mostly grouped. 

To make the model compatible to the GEORES plugin of sketch up, the layer need to named according to some rules which will be explained in the following chapter.
All layers belonging to one building need to start with the same prefix, followed by a dot.
\begin{lstlisting}
building1.
\end{lstlisting}

For each wall, ground or roof surface an extra layer has to be created as subitems of the building and the given ids of the surfaces have to unique within the model.
\begin{lstlisting}
building1.wall1
building1.roof1
building1.ground1
\end{lstlisting}
 Surfaces belonging to one type can also be assigned to one layer, but they need to be grouped instead of the id,  a "s" indicates that there are grouped surfaces in the layer.  If a surface has subsurfaces (openings) like windows or doors it cannot be grouped and needs to be in an extra layer as shown above.
 \begin{lstlisting}
  building1.walls
building1.roofs
building1.grounds
 \end{lstlisting}

Openings must be either a door or windows and the layer name follows the same schema.
A unique id for each opening, or openings belonging to one surface can be grouped and assigned to one layer ending on "s".

\begin{lstlisting}
 building1.wall1.window1
building1.wall1.window2
building1.wall1.door1
building1.wall2.windows
building1.wall2.doors
\end{lstlisting}

\subsubsection{Geocoding and export to cityGML}

Geo location is the alignment of the building data to a known reference system, in order to represent the model in the true location in real space.  Georeferencing is very important in modeling and using the right reference coordinate system is very important so that the integration of the model can be done in the city model for compatibility and consistence reason. For the georeference of the building model the following steps have to be followed.
First of all the building has to be rotated. Therefore the directional angle of on side of the corresponding LOD2 is taken, to be able to rotate the new LOD 3 model in the right way. The position of the building is added during the export to cityGML using the GEORES Plugin for Sketch Up. Within the dialog box of the export the shift in east and north direction has to be filled in.

\subsubsection{Adding of generic attributes to the building} 

This is the phase where attributes are added to the model, using the FME software application. Different kinds of attributes are added to each building model, the generic attributes, the standard attributes, wall attributes and so on.  Different kinds of properties or instances are added to the building model as attributes. These properties vary from general information like building age, number of stories, energy related properties like u values and the wall to window ratio.

The U values for each building part are obtained from the Energieprofil Kurzverfahren (annex 3) which categorize the materials used. The U-value for the materials are based on the building age class. The u value is also known as the thermal transmittance which is the energy (W) transmitted through one $m^2$ of a material with a certain temperature difference between both sides. The unit of this value is [W/ (m²K)].
The window to wall ratio is computed and added as an attributes also. In the Energieprofil Kurzverfahren there is a reference for building classes and the window to wall ratio.

As mentioned above the software FME has been used to add the attributes to the building. To do so, following workflow has been developed by the group of students.

\begin{itemize}
\item creation of a writer
\subitem used format: City GML
\subitem dataset: the exported Citygml file
\subitem Workflow Options: Static Schema
\item Create a new Reader
\subitem select City gml as file
\subitem select your exported City GML file
\subitem select individual feature types
\subitem click ok
\subitem check all boxes
\subitem check also add to writer
\subitem click ok
\item Delete the writer you created first
\item Open the Feature Type Properties of the writer of a Surface you like (e.g. Wall Surface)
\subitem go to User Attributes
\subitem add a new name and chose citygml string as format
\subitem click ok
\item Double click the red arrow in front of the new added attribute
\subitem type in the value of the attribute
\subitem Set coordinate system
\subitem expand writer in navigator
\item set coordinate system to "DHDN.Berlin/Cassini"
\subitem expand "Parameters"
\subitem set "GML srsName" to "3068"
\subitem set "GML SRS Axis Order" to "1,2,3"
\item Click run
\end{itemize}

Following attributes have been created using this approach.
\begin{itemize}
\item Building
\subitem volume
\subitem height
\subitem no\_of\_floors
\subitem no\_of\_flats
\subitem total\_wall\_area
\subitem adjacentWall\_wall\_area
\subitem no\_adjacentWall\_wall\_area
\subitem roof\_area
\subitem window\_area 
\subitem heating\_system
\subitem no\_of\_inhabitants
\subitem basement (boolean)
\subitem ratio\_windowArea\_totalWallArea
\subitem ratio\_windowArea\_totalAreaWallWithWindows
\subitem ratio\_windowArea\_noAdjacentWallArea 
\subitem ratio\_windowArea\_floorSpace (floorspace = ground\_area * storeys)
\subitem citygml\_class
\subitem citygml\_function
\subitem citygml\_usage
\subitem citygml\_year\_of\_construction
\subitem citygml\_storeys\_above\_ground
\subitem citygml\_storeys\_above\_ground
\item Wall Surfaces
\subitem u value
\subitem material
\subitem thickness
\item Roof Surfaces 
\subitem u\_value
\subitem materials
\item Windows
\subitem u\_value
\end{itemize}



\subsubsection{Validation of City GML file and integration into the 3D city database}

The validation and integration was done by using the Importer-Exporter Tool developed by our institute. After setting up the connection to the 3d- city database the City GML file with the new Building has to be selected under the tab Import. Every setting is left on default setting. The validate button is clicked and the validation is done automatically starts the validation of the input file against the xml schema definition of the City GML application schema. After the validation the file is suitable for the import.  The button import is then pressed to integrate the model in the database.

\subsubsection{Visualization and presentation of model }

The visualization is done using goggle earth, the four different building are viewed in there different location in space. 
The results of the building model were presented below with all the major details of the models in sketch up after additional of all the attributes and projection of textures. Figure \ref{fig:kaiserin} shows one of the four building which have been modeled.
\begin{figure}[ht]
	\centering
	\includegraphics[width=0.8\textwidth]{phase1/group2/figures/kaiserin.png}
	\caption{Model of building located at Kaiserin-Augusta-Alle 10B}
	\label{fig:kaiserin}
\end{figure}


\subsection{Conclusion}
From the visualization of the results of the building model, it can be seen that the building model created had been used to replace the LoD 2 model, with this change the city gml model of berlin has been updated for four selected buildings to LoD 3 with useful energy attributes and generic attributes that can allowed for different query and simulations. The approach has seen compared to previous approach can be said to be cost effective and possible for implementation on small data set. This approach will not be suitable on large set of data because it might be very complicated and time consuming. It will be ideal to use this project approach for regular or periodic update of selected data or addition of new data of newly constructed building into the 3D cadastral. 






\chapter{
\section{Group 3 - Further Sources for Additional Data and Integration on 3D Models}
}%group 3
\input{phase1/group3/report_ph1_g3}


\part{Estimation of Energy Consumption and Energy Demand}
\newpage

\chapter{ %group 1
\section{Group1 - Calculation of buildings energy demand}
}
\subsection{Introduction}
The Task for the second phase was to calculate the building's energy demand using Java with the citygml4j library and the 3D city model. Necessary for that are the building’s volume and the attribution of walls as outer- or inner-/shared- wall surfaces which already was calculated by group one in phase one.\\
The most important information for this task are content of the so called IWU Report which is a report created by the IGG (Institute for Geodesy and Geoinformation Science) based on calculation methods of the "Institut Wohnen und Umwelt GmbH" [eng. Institute living and environment GmbH]. The IWU Report can be interpreted as a manual for the calculation of the building energy demand using the semantic citymodel of Berlin.\\
The document is structured into three parts: The determination of the input-values which are temperature, geometries and energy reference building-parameter, the calculation of the building's energy demand for space-heating using an accounting system and the calculation of the building's energy demand for warm-water.

\subsection{Input values}
For the calculation of energy flow of buildings, the surrounding climate is of an big importance. To take this into account some formulas contain a variable called "Gradstunden" which is representing the climate in a numerical value. For its calculation the summed up days per year have to be multiplied with the difference of the heated temperature inside and the actual temperature outside. To neglect the summer season in which buildings do not need any external heating energy, only the days with a switched on heater were considered.\\
The energy reference area is relevant for all calculations dealing with the building size. And due to the fact, that a precise indoor-model of the city is not available, the estimation of this value can be done with the formula (\ref{eq:energyReferenceArea}). Important to know is that only heated storeys without roof and cellar are taken into account.
\begin{equation}
	A_{EB} = 0.75 \times n_{G} x A_{FB} [m^2]
	\label{eq:energyReferenceArea}
\end{equation}
\begin{description}
	\item[$n_{G}$:] number of storeys
	\item[$A_{FB}$:] building footprint
\end{description}
Necessary for air-volume calculations inside, either the precise storey height have to known, or the average storey height which is depending on type and age of the building. Figure \ref{fig:storeyHeight} shows some average values of storey height depending on the building's age. Furthermore we took for our calculations additional 0.3m per storey in order to achieve (more or less) the same number of storeys as in real world. This number was found be supervising some samples of buildings which are part in our database-subset.
\begin{figure}[h]
	\centering
 	 \includegraphics[scale=0.5]{phase2/group1/storey_height.jpg} 
	\caption{Average storey height depending on building's age. Provided by Michael Prytula.}
	 \label{fig:storeyHeight}
\end{figure}
The air-volume then can be calculated with the formula (\ref{eq:airVolume}) which is the energy reference area multiplied with the assumed mean (lighted) storey height.
\begin{equation}
	V_L = A_{EB} \times 2.5
	\label{eq:airVolume}
\end{equation}
Since some energy flows through building parts, its physical behaviour has be taken into account when calculation the energy flow. For this issue the so called U-values (in the IWU-Report called "k-Werte") are needed. They are describing the amount of energy transmitting between different building parts as for example between building-roof and building-main-part. Due to the fact that energy will not flow between two parts having the same temperature, it is important to know which wall surfaces are shared with also heating neighboured buildings. Figure \ref{fig:uValues} shows possible directions of the energy transmission, and possible building parts differentiated by its mean temperature. The afterwards following figure \ref{fig:uValuesTable} shows average U-values for specific building-ages and building-parts.
\begin{figure}[h]
	\centering
 	 \includegraphics[scale=0.4]{phase2/group1/U-Values.jpg}
	\caption{Possible energy flow directions of a building.}
	\label{fig:uValues}
\end{figure}
\begin{figure}[h]
	\centering
 	 \includegraphics[scale=0.59]{phase2/group1/u-values_table.jpg}
	\caption{Average U-Values for specific building-ages and building-parts. Provided by Michael Prytula.}
	\label{fig:uValuesTable}
\end{figure}

\subsection{Demand of effective energy for heating}
The following formulas are describing the sequence of calculating the effective energy demand for heating. Later on following formulas are partially depending on the previous ones.\\
The calculation of the energy demand for heating can be easily done with using an accounting system as follows (\ref{eq:heatingDemand}):
\begin{equation}
Demand = Gain - Loss
\label{eq:heatingDemand}
\end{equation}
The Energy loss in [kWh/a] can be calculated as follows (\ref{eq:energyLoss}). 
\begin{equation}
Q_{V} = (Q_T + Q_L) \times f_{Abs}
\label{eq:energyLoss}
\end{equation}
\begin{description}
\item[$Q_{T}$:] transmission loss [kWh/a]
\item[$Q_L$: ]aeration loss [kWh/a]
\item[$f_{Abs}$:] reduction factor day-/night-setback
\end{description}
The values for the specific Day-nightsetback can be taken from a table with average values, depending on the building type (e.g. new-building).\\
The loss through transmission in [kWh/a] has to be taken from this formula (\ref{eq:lossTransmission}):
\begin{equation}
Q_T = (\sum f_i \times k_I \times A_i) \times \Theta 
\label{eq:lossTransmission}
\end{equation}
\begin{description}
\item[$f_i$:]reductionfactor [1.0 outer-walls; 0.5 inner-walls]
\item[$k_I$:] U-value [W/($m2$K)]
\item[$A_i$:] building's area [$m^2$]
\item[$\Theta$:] Gradstunden [kKh/a]
\end{description}
Summed up for every building part [i].\\
The loss through aeration in [kWh/a] can be calculated using the formula (\ref{eq:lossAeration}):
\begin{equation}
Q_L = 0.34 \times n \times V_L \times \Theta 
\label{eq:lossAeration}
\end{equation}
\begin{description}
\item[$n$:] frequency of aeration [1/h] (from table)
\item[$V_L$:] building's air volume [$m^3$]
\end{description}
For the energy gain following formulas have to be applied:\\
In general formula (\ref{eq:energyGain}) is describing the utilisation of the available free heat inside of the building:
\begin{equation}
Q_G = \eta_F \times Q_F 
\label{eq:energyGain}
\end{equation}
\begin{description}
\item[$\eta_F$:] utilisation
\item[$Q_F$:] free heat
\end{description}
The free heat is simply the solar irradiation plus the heat sources inside of the building (see formula (\ref{eq:freeHeat})).
\begin{equation}
Q_F = Q_S + Q_I 
\label{eq:freeHeat}
\end{equation}
\begin{description}
\item[$Q_S $:] solar irradiation
\item[$Q_I $:] heat sources inside
\end{description}
And following formula (\ref{eq:utilisation}) can be used for the utilisation:
\begin{equation}
\eta_F = 1 - 0.3 \times (\dfrac{Q_F}{Q_V}) 
\label{eq:utilisation}
\end{equation}
\begin{description}
\item[$Q_V $:] (remember:) energy-loss
\end{description}
As important income the solar energy gain can be calculated with the precise formula (\ref{eq:solarGain}) which needs precise knowledge about the window sizes.
\begin{equation}
Q_S = r \times g_{senkr} \times \sum G_i \times A_{F,i} 
\label{eq:solarGain}
\end{equation}
\begin{description}
\item[$G_i $:] global radiation per orientation (e.g. south)
\item[$A_{F,i} $:] window area per orientation [$m^2$]
\item[$g_{senkr} $:] energy transmission through glass-area (from Kurzverfahren Energieprofil)
\item[$r $:] reduction factor due to windows (standard value: 0.36)
\end{description}
Taking an factor for the window sizes per wall, also the following simplified formula (\ref{eq:simplySolarGain}) can be used for the calculations.
\begin{equation}
Q_S = r \times g_{senkr} \times 240 \times A_{window} 
\label{eq:simplySolarGain}
\end{equation}
\begin{description}
\item[$A_{window} $:] estimated overall window area [$m^2$]
\end{description}
Inside of buildings there are more heat sources than only the heater. For example electric devices like the light bulb are producing a lot of heat energy, which is also a factor for the calculations.\\
As a assumption, the next formula (\ref{eq:innerHeatSources}) is giving this factor a size.
\begin{equation}
Q_I = 0.024 \times q_i \times t_H \times A_{EB} 
\label{eq:innerHeatSources}
\end{equation}
\begin{description}
\item[$0.024 $:] factor for conversation ([W] $\rightarrow$ [kW]; [d] $\rightarrow$ [h]
\item[$q_i $:] specific power of inside heat sources [W/$m^2$]
\item[$t_H $:] heating period [d/a]
\item[$A_{EB} $:] energy reference area
\end{description}

\subsection{Demand of effective energy for warmwater}
The demand of warm water [kWh/a] can more easily be calculated. It is simply the demand per person multiplied with the number of persons living in the building. The formula (\ref{eq:warmWater}) shows how this can be calculated.
\begin{equation}
Q_W = \dfrac{Q_{W/P} \times A_{EB}}{A_{EB/P}}
\label{eq:warmWater}
\end{equation}
\begin{description}
\item[$Q_{W/P} $:] demand of warm water per year and person [kWh/(P a)] (standard: 600 kWh/(P a))\\
\item[$A_{EB/P} $:] living space per person [$m^2$/P] (standard: 35 $m^2$/P)\\
\end{description}
This formula calculates the number of persons living in the building by dividing the energy related area by the average space per person. When having a more precise estimation of inhabitants per building as group three was calculating, the formula can be changed into the formula (\ref{eq:simplyWarmWater}):
\begin{equation}
Q_W = Q_{W/P} \times N_P
\label{eq:simplyWarmWater}
\end{equation}
\begin{description}
\item[$N_P$:] number of persons\\
\end{description}

\subsection{Result}
As results of the calculations, the Java algorithm produces a .csv file containing information about each building which has an own identifier. All values are calculated in [kWh/a]. The table shows a subset of the result file, with three buildings and the interesting attributes described in this report.
\begin{table}[b]
\centering
\begin{tabular}{c  c  c  c  c  c}
building\_id & heating\_loss & heating\_gain & ... & ..demand\_heating & ..demand\_warmwater \\
\hline						
BLDG\_000300000026ed79 & 116670,763	& 15572,718 & ... & 101098,045 & 4805,405\\
BLDG\_000300000026f491 & 229314,037	& 32343,216 & ... & 196970,821 & 16289,207\\
BLDG\_000300000026f4a7 & 216882,225	& 31687,923 & ... & 185194,302 & 15044,102\\
...\\
\end{tabular}
\label{table:result IWU-Report}
\caption{Results of the IWU-report-calculation [kWh/a].} 
\end{table}

\subsection{Discussion}
The number of storeys is one central value of all calculation, because with a wrong estimated number of storeys the reference are can change rapidly. Figure \ref{fig:changeArea} shows how the reference area is changing with the wrong estimated number of storeys.
\begin{figure}[h]
	\centering
 	 \includegraphics[scale=0.5]{phase2/group1/change_area.png} 
	\caption{Changing area with wrong estimated number of storeys.}
	 \label{fig:changeArea}
\end{figure}
And therewith also the calculated energy parameters which are depending on the reference area are changing.  Figure \ref{fig:changeParameter} shows how strong they are depending on a right estimated number of storeys.
\begin{figure}[h]
	\centering
 	 \includegraphics[scale=0.5]{phase2/group1/change_param.png} 
	\caption{Changing energy parameters with wrong estimated number of storeys.}
	 \label{fig:changeParameter}
\end{figure}

Following improvements could be helpful to obtain better results:\\
First as mentioned before we took additional 0.3m to obtain a better estimation, this is because in the calculations the average storey height was describing the inside storey height without ceiling thickness.\\
Maybe it would be useful to consider the building's usage for the estimation of the specific storey height.\\
In the IWU report 2.5m were assumed as the storey height for the air-volume, in our calculations we were using the previously calculated storey height.


\newpage

\chapter{ %group 2
\section{Group 2 - Estimation of the Solar Potential}
}
%%**************************************************************
%% GIS project report
%%**************************************************************

% possible structure
% x Estimation of the solar Potential with 3D CityDB (Introduction) DONE
% x.2 Workflow (Steffen) DONE
% x.3 Basics of PV systems (Hannah)DONE
% x.4 Basics of ST systems (Steffen) DONE
% x.5 Solar irradiance (Hannah) DONE
% 	x.5.1 Tilt (Hannah) DONE
% 	x.5.2 azimuth (Hannah) DONE
% 	x.5.3 satel light db (Hannah) DONE
% x.6 Roof surface area DONE
% 	x.6.1 simple reduction (Steffen) DONE
% 	x.6.2 reduction due to building age classes (Steffen) DONE
% x.7 Shadowing (Steffen) DONE
% x.8 Calculation of the potential DONE
%	x.8.1 Photovoltaic (Hannah) DONE
% 	x.8.2 Solar Thermal (Steffen) DONE
% x.9 Results (Steffen)
%	x.9.1 Test Area (Steffen)
% 	x.9.2 Validation (Steffen)
% x.10 Visualization (Hannah)
% x.11 Conclusion (Hannah) First version is DONE
% x.12 Future Work (Hannah) DONE


%\documentclass[10pt,a4paper,portrait]{article}
\usepackage{geometry}                % See geometry.pdf to learn the layout options. There are lots.
\geometry{a4paper}                   % ... or a4paper or a5paper or ... 
%\geometry{landscape}                % Activate for for rotated page geometry
\usepackage[parfill]{parskip}    % Activate to begin paragraphs with an empty line rather than an indent
\usepackage{graphicx}
\usepackage{subfig}
\usepackage{amssymb}
\usepackage{amsmath}
\usepackage{epstopdf}
\usepackage{multicol}
\usepackage{amsmath}
\usepackage{listings}
\usepackage{url}
\usepackage{multirow}
\usepackage{color}
\definecolor{grey}{rgb}{0.5,0.5,0.5}
\definecolor{darkgreen}{rgb}{0.5,0.5,0.0}
\lstset{ %
language=Matlab, % the language of the code
basicstyle=\ttfamily, % the size of the fonts
       % that are used for the code
numbers=left, % where to put the line-numbers
numberstyle=\footnotesize, % the size of the
        %fonts that are used for the line-numbers
stepnumber=1, % the step between two
       % line-numbers. If it's 1, each line 
keywordstyle=\color{darkgreen}, % Keywords
       % font ('*' = uppercase)
commentstyle=\color{grey},
numbersep=5pt, % how far the line-numbers are
       % from the code
backgroundcolor=\color{white}, % choose the
        %background color. You must add \usepackage{color}
showspaces=false, % show spaces adding
        %particular underscores
showstringspaces=false, % underline spaces
        %within strings
showtabs=false, % show tabs within strings
       % adding particular underscores
frame=single, % adds a frame around the code
tabsize=2, % sets default tabsize to 2 spaces
captionpos=b, % sets the caption-position to
        %bottom
breaklines=true, % sets automatic line
        %breaking
breakatwhitespace=false, % sets if automatic
        %breaks should only happen at whitespace
 % show the filename of files included with
       % \lstinputlisting;
 % also try caption instead of title
escapeinside={\%*}{*)} % if you want to add a
       % comment within your code
}
\DeclareGraphicsRule{.tif}{png}{.png}{`convert #1 `dirname #1`/`basename #1 .tif`.png}
% 
% \begin{document}
% 
% % Titlepage with task representation
% \input{include/title}	
% \newpage
% 	

% Introduction + workflow
\chapter{Einführung in die Problematik}
Diese Masterarbeit beschreibt den potentiellen Nutzen eines grafischen Management Systems für die Gebäudeüberwachung mittels verschiedener stationärer und mobiler Sensoren. Anhand eines Beispieles soll geklärt werden welche Technologien eingesetzt werden müssten um diese Aufgabe zu lösen, und worin genau der Mehrwert eines solchen Systems gegenüber konventioneller Vermessungssoftware liegt. Die vorgeschlagenen Technologien werden zum Abschluss der Arbeit teilweise praktisch angewandt respektive umgesetzt. Dieser Prototyp soll exemplarisch demonstrieren wie solch ein System arbeiten wird. Kernfragen der Masterarbeit ist, wie kann dieses System Feldingenieuren, die im Bereich der Gebäudeüberwachung arbeiten, generell in den folgenden Bereichen unterstützen:
\begin{itemize}
\item Planung und Optimierung der allgemeinen Arbeiten im Feld
\item Datenkommunikation mit dem Büro
\item Vorabauswertung der Messwerte direkt im Feld
\end{itemize}
Die Klassische Arbeit von Vermessungsingenieuren besteht aus dem praktischen Teil der im Feld durchgeführt wird, und den anschließenden Auswertungen und der Interpretation im Büro. Durch die räumliche Trennung dieser Beiden dennoch sehr ineinander verzahnten Aufgaben entsteht oftmals eine Verzögerung in den Abläufen und erhöhtes Risiko für vermeidbare Fehler in den Arbeitsabläufen. Das System soll so konzeptioniert sein, dass es die Lücke schließt, zum Einen um die Effizienz der Arbeiten zu erhöhen, und zum Anderen um Fehler bei den Messungen oder der Datenmigration zu erkennen oder von vorn herein zu vermeiden. 

Heutzutage ist das Bearbeiten von unterschiedlichen Arbeiten an einem Ort zur gleichen Zeit keine Vision mehr. Mobile Endgeräte wie \"Smartphones\" oder \"Tablet-Computer\" vereinfachen Arbeitsabläufe, und helfen Zeit zu sparen. Bei den Arbeiten im Feld sind mobile Endgeräte bereits ständig präsent, dennoch helfen sie lediglich bei wenigen Aufgaben wie der papierlosen Bürokratie, der Email-Kommunikation mit dem Büro oder den betrachten vorheriger Messerergebnisse. Das hier konzipierte System hingegen weißt folgende drei Hauptvorteile gegenüber der aktuellen Nutzung von mobilen Endgeräte auf:
\begin{itemize}
\item Kernfunktionalität ist das Assistieren des Vermessungsingenieurs. Das heißt es soll Hilfestellung beim Verstehen der Messungen geben (zum Beispiel durch den Vergleich der aktuellen Messungen mit vergangenen Messreihen). $\rightarrow$ Das Teilen von Informationen zwischen Feld und Büro führt zu einem besseren Kenntnisstand während der Arbeiten im Feld.
\item Fehlmessungen wie sie etwa beim vertauschen von Positionen geschehen sollen vermieden werden indem die Messergebnisse direkt nach Eingabe in das System auf ihre Konsistenz hin überprüft werden.
\item Für die Aufnahmen im Feld brauchen vorherige Messreihen nicht umständlich exportiert zu werden, sondern diese können von dem Mobilen System direkt von dem gemeinsamen Daten-Server abgerufen werden.
\end{itemize}
Das hier konzeptionierte System ist keine Alternative für klassische Vermessungssoftware, ist es eine Ergänzung und eine Brücke von den Sensoren direkt zu dem System. Es ermöglicht die Kombination von klassischer Ingenieurvermessung mit Sensornetzwerken.


\section{Vergleichbare Systeme}
Das Konzept der vernetzten Sensoren wurde bereits in einigen Systemen erfolgreich angewandt. Gerade bei Problemen die sich auf einen größeren Raum und über eine längere Zeit erstrecken ist es häufig nicht mehr möglich mit Einzelmessungen genug Informationen für eine Analyse zu erhalten. Ein wichtiges Argument für den Einsatz solcher Netzwerke ist natürlich auch die Möglichkeit alle verbundenen Sensoren bequem und ohne den Einsatz zusätzlicher Mittel von einem Ort aus kontrollieren zu können. Fehler in der Software beeinflussen damit natürlich auch das gesamte System, die Fehlersuche beschränkt sich dann aber auch nur auf einen Punkt. Nicht zuletzt spielt auch die Ökonomische Betrachtung solcher Systeme eine wichtige Rolle, Zentralisierung von Kräften bedeutet auch eine REduzierung von Kosten. Wenige Spezialisten ersetzen eine große Anzahl an Generalisten im Feld. Eine tiefer gehende Diskussion von Sensornetzwerken im allgemeinen und des Prototypen im Speziellen ist im dritten Teil der Arbeit zu finden.

\begin{wrapfigure}{r}{0.5\textwidth}
  \begin{center}
 	 \includegraphics[scale=0.45]{graphics/GITEWS_Warning_Centre_01.jpg} 
	\caption{Tsunami Warnzentrum in Indonesien, GITEWS 2011}
  \end{center}
\end{wrapfigure}
Ein sehr prominentes Beispiel für den Einsatz von Sensornetzwerken ist das Deutsch-Indonesische Tsunami Frühwarnsystem \newacronym{GITEWS}{GITEWS}{German Indonesian Tsunami Early Warning System} \gls{GITEWS} Nach dem verheerenden Tsunami von 2006 im Pazifischen Ozean, der vor allem in Indonesian viele Opfer forderte, beschloss Deutschland den Aufbau eines Frühwarnsystems, dass die Menschen in Indonesien besser vor Tsunamis warnen sollte. 
Tsunamis werden meist durch eine spezielle Art von Erdbeben ausgelöst, bei der sich der Seeboden in vertikaler Richtung bewegt und damit Wasser verdrängt. Dadurch entsteht ein Berg aus Wasser, der dann in alle Richtungen ausläuft. Somit können Tsunamis vorhergesagt werden, indem alle vorkommenden Seebeben daraufhin untersucht werden ob und wo ein Tsunami entstehen könnte. Das \gls{GITEWS} System basiert auf einem dichten Netz aus Seismographen, \newacronym{GPS}{GPS}{Global Positioning System} GPS-Stationen und Pegelstationen rund um Indonesien. Seismographen an sich können zwar Erdbeben registrieren, sind aber alleine nicht in der Lage Aussagen über Epizentrum, Art und Stärke des Erdbebens zuzulassen. Die Vernetzung der Stationen stellt demnach eine der essentiellen Eigenschaften eines Erdbebenwarnsystems dar. Und damit die Warnung früh genug verbreitet werden kann müssen die Daten innerhalb eines äußerst kurzen Zeitraums erfasst und ausgewertet werden. Am besten gelingt das indem die Daten in Echtzeit übermittelt werden, und kontinuierlich ausgewertet werden. Das System dient als Kommunikations-Knoten zwischen allen Sensoren, als Auswertungssystem für die Daten und auch als Visualisierungsplattform für die Ergebnisse. \citep{lauterjung_gitewstsunami-fruhwarnsystem_2011} \citep{strobl_geodatenmanagement_2007} \citep{spahn_experience_2010}

In dem Projekt \newacronym{SOSEWIN}{SOSEWIN}{Self-Organising Seismic Early Warning Information Network}\gls{SOSEWIN} beschäftigen sich Wissenschaftler ebenfalls mit einem Netzwerk aus Seismometern und weiteren Messinstrumenten. Bei diesem Projekt sollen jedoch nicht vor Tsunamis, sondern vor Erdbeben im Raum Istanbul gewarnt werden. 

\begin{wrapfigure}{r}{0.5\textwidth}
  \begin{center}
 	 \includegraphics[scale=0.33]{graphics/SOSEWIN_Sensor_Bridge.jpg} 
	\caption{Sensor an der Sultan Mehmet-Brücke über den Bosporus, SOSEWIN 2011}
  \end{center}
\end{wrapfigure}
Dazu werden engmaschig Seismometerstationen und zudem im Stadtgebiet Bewegungssensoren  und GPS-Empfänger installiert. Aus der Kombination der verschiedenen Messwerte können einige Sekunden vor einem Erdbeben Warnungen ausgegeben werden. Dieses Sensornetzwerk stellt ein sich selbst organisierendes und über TCP/IP kommunizierendes Netzwerk dar. Anders als bei anderen Netzwerken organisiert nicht ein zentraler Knoten alle Sensoren, sondern die Sensoren sind jeweils mit einem Mini-Computer verbunden, der dann aktiv mit den anderen Knoten kommuniziert. Geplant ist auch eine Erweiterung des bestehenden Netzwerkes indem Privathaushalte Sensorkomponenten erwerben, und sich dadurch aktiv an der Frühwarnung beteiligen. Diese Möglichkeit besteht nur weil die Sensoren "quasi-autonom" funktionieren. Zusätzlich sollen auch kritische Infrastrukturen überwacht werden. Testweise wurden bereits Instrumente an der Sultan Mehmet-Brücke über den Bosporus installiert um deren charakteristischen Eigenschwingungen zu messen. Gerade diese Brückenüberwachung hat gewissen Ähnlichkeit mit dem hier konzipierten Beispielnetzwerk, das zu Grunde liegende System allerdings unterscheidet sich maßgeblich. \citep{luhr_sekunden_2011}


\chapter{Softwareanforderungen}
Die Konzeptions eines Systems muss den Anforderungen der Nutzer entsprechen, und darf nicht an dem bestehenden Bedürfnissen "vorbei geplant" werden. Die Anforderungen an das System ergeben sich aus technischen Anforderungen, Erwartungen potentieller Nutzer und die damit einhergehenden Beispielfunktionen des Systems. Der Aufbau des Anforderungskataloges besteht aus einer Beschreibung des Einsatzgebietes, einer Beschreibung des Systems und den daraus abzuleitenden funktionalen und nicht-funktionalen Anforderungen.


\section{Einsatzgebiet}
Um präzise beschreiben zu können was das System tun können muss, ist es notwendig vorher konkret die vorliegende Situation oder das existierende Problem zu definieren. Solche eine Definition sollte eine Beschreibung der Umgebung beinhalten in der das System eingesetzt werden soll. Nimmt man eine Modellierungssprache zur Hilfe, um die Umgebung zu beschrieben, ist es später einfacher daraus Rückschlüsse auf mögliche Probleme zu ziehen. Für ein solches Modell müssen die Fragen beantwortet werden, in welche Objekte sich die Umgebung aufgliedert, wie diese interagieren und welche Funktionalitäten diese dafür verwenden.Aber auch die teilhabenden Akteure und ihre konkreten Anforderungen an das System müssen modelliert werden. Mit Hilfe von exemplarischen Anwendungsfällen werde ich beschreiben was der tatsächliche Bedarf des Nutzers ist, und wie dieser gedeckt werden kann.

Bevor ich mit dem technischen Ausformulieren der Modellierung beginne, möchte ich kurz in die Thematik einleiten: Das System welches ich mit dieser Arbeit konzeptioniere soll Feldingenieuren helfen im Feld mit den Messungen und Daten verschiedener Sensoren zu arbeiten. Als konkretes Beispiel werde ich den Einsatz des Systems bei der Bauwerksüberwachung mithilfe eines Sensornetzwerkes beschreiben.

Die Überwachung von Bauwerken mittels eines Netzwerkes aus verschiedenen Sensoren hilft ihre Sicherheit ohne den Einsatz großer Bautechnischer Überprüfungen einschätzen zu können. Damit ist es möglich Bauwerke auch weit über ihre geplante Lebensdauer hinweg zu erhalten. Ohne den Einsatz solcher Sensor Netzwerke können die zuständigen Gutachter bei Ablauf der geplanten Lebensdauer nicht darauf vertrauen, dass das Gebäude auch weiterhin den kontinuierlichen Belastungen gewachsen ist, und somit werden entweder umfangreiche Sanierungen Nötig, oder Gebäude werde geschlossen. Die Überwachung basiert auf der Messung von Veränderungen von verschiedenen Parametern wie zum Beispiel der Position , der Temperatur oder Feuchtigkeit von Bauteile oder der Abweichungen von charakteristischen Bewegungsmustern von Bauteilen, gemessen durch Beschleunigungssensoren. Die Parameter werden sowohl punktuell  verteilt über das gesamt Bauwerk, als auch gesamtheitlich die Struktur des Bauwerkes miteinbeziehend erhoben, siehe auch \citep{farrar_introduction_2007} \citep{worden_overview_2004} \citep{boller_structural_2004}Für die Messungen werden zum einen automatisch kontinuierlich messende Systeme eingesetzt, und zum anderen seltenere manuelle Messungen, deren Ergebnisse manuelle in das System eingegeben werden. 

Für das bessere Verständnis möchte ich hier ein Beispielfall beschreiben: Ein Brücke erreicht ihre letzten Jahre der Betriebserlaubnis. Danach müssen entweder die Verkehrssicherheit erneut umfangreich geprüft, und zahlreiche Verschleißteile, deren Zustand schlecht zu beurteilen ist, ersetzt werden, oder die Verkehrssicherheit muss auf andere Art überprüft werden. Zahlreiche Sensoren werden an den einzelnen wichtigen Gebäudeteilen eingerichtet, und überwachen nun automatisch über einen bestimmten Zeitraum deren Verhalten und Veränderungen. In periodischen Abständen werden automatisch Diagnosen erstellt, basierend auf der Analyse der Messwerte, der Überprüfung des Materialverschleißes und einiger anderer Einflussgrößen. Eine detailliertere Beschreibung der verwendeten Messungen, Zeitskalen und Analysemethoden werden in den nachfolgenden Kapiteln beschrieben.

In der Einleitung der Arbeit möchte ich mich am Verlauf der Erstellung eines Pflichtenheftes für die Softwareentwicklung orientieren , da so am besten modelliert werden kann wie der Bedarf des Nutzers gedeckt werden kann, siehe auch \citep{engels_vorlesung_2006}. Beginnen werde ich mit einer textuellen Beschreibung der Situation. Danach folgt eine Modellierung der Prozesse und der Akteure mit ihren jeweiligen Anwendungsfällen. Zum Abschluss werde ich dann die daraus abgeleiteten notwendigen Funktionalitäten des System beschreiben.


\subsection{Modell des Problembereichs}
Die Abbildung \ref{fig:model_domain} zeigt ein \newacronym{UML}{UML}{Unified Modeling Language} \gls{UML} Diagramm das die im folgenden beschrieben verschiedenen Objekte des Systems beinhaltet. Das Modell beschreibt die Beziehungen der einzelnen Objekte untereinander und modelliert keine Aktivitäten oder Funktionen.

Die Umgebung in der das System eingesetzt werden wird besteht aus fünf verschiedenen Arten von Objekten und deren Beziehungen untereinander. Zentrales Objekt ist der Daten Server, der als Knoten für die Kommunikation zwischen den einzelnen Kompartimente dient. Diese sind hauptsächlich die Sensoren selbst, die jedoch ohne einen Server, der als Steuerungseinheit für jeden Sensor dient, nicht selbständig messen können. Der Server kontrolliert die Sensoren indem er sie aktiviert und deaktiviert. Nichtsdestotrotz können Sensoren in einem separiertem eigenem Netzwerk organisiert sein, das dann wiederum als einzelner Sensor behandelt wird. Die Sensoren senden ihre gemessenen Daten entweder aktiv an den Server beziehungsweise über den Server an die dem Server angeschlossene Datenbank, oder der Server ruft die Daten aktiv ab, und speichert diese dann in der Datenbank.

\begin{figure}[H]
	\centering
 	 \includegraphics[scale=0.6]{graphics/model_of_issue.jpg} 
	\caption{Modell des Problembereiches mit relevanten Objekten, F. H. Euteneuer 2013}
	 \label{fig:model_domain}
\end{figure}

Die Datenbank die and den Server angeschlossen ist speichert sowohl Metadaten zu den Sensoren, als auch die gemessenen Werte. Unter Metadaten sind alle Informationen zu verstehen, die die Sensoren eindeutig beschreiben, und die für weitere Analysen der Messerwerte erforderlich sind (siehe auch im Glossar "Metadaten". Beispielsweise sind das die Positionen der Sensoren, die Messintervalle, die Sensortypen oder die übermittelten Datentypen.

Als Klient des Services kann im Prinzip jede Art von mobilen Systemen eingesetzt werden. Angeschlossen an die Datenbank dienen diese dann als bildgebender Teil des Systems. Da die Verknüpfung mit einem Server meist über das Protokoll \newacronym{TCP/IP}{TCP/IP}{Transmission Control Protocol/Internet Protocol} \gls{TCP/IP} geschieht, müssen mobile Geräte über eine Internetverbindung verfügen. Die Verwendung dieser Geräte bleibt dadurch begrenzt auf Gebiete innerhalb der Handynetz-Abdeckung. Für manuelle Messungen dient der mobile Klient zusätzlich als Eingabegerät für die Messwerte, sofern dies nicht über das Gerät selber erfolgen kann. Dadurch wird der mobile Klient in dem Modell sowohl als bildgebender Teil des Systems, als auch als Sensor behandelt, und ererbt damit die Eigenschaften des Sensor Objekts.

Das System will einen ganzheitlichen Ansatz verfolgen, und beinhaltet somit auch einen Teil der für die umfangreicheren Analysen zuständig ist, sowie durch eine Datenexportfunktion als Schnittstelle zu weiteren Algorithmen und System dient. Dieser Teil des System wird in dem Modell durch den "Desktop-Computer" repräsentiert. Die eigentliche Einrichtung und Planung des Systems wird erwartungsgemäß von diesem, dem bequemeren Arbeitsplatz (verglichen mit dem mobilen Klienten), durchgeführt werden. Zusätzlich zu den Eigenschaften des Feldcomputers sind somit erweiterte Verwaltungs- und Analysefunktionen als Eigenschaften dieses Objektes definiert.


\subsection{Geschäftsprozesse}
Wichtigste Entscheidungshilfe für die Nutzung solch eines Systems wird die Eigenschaft des Systems, eine entscheidungsunterstützende Funktion zu erfüllen, sein. Das System ist fokussiert auf die wichtigen Werkzeuge die die Arbeit des Feldingenieurs vereinfachen sollen, und lässt unwichtige oder komplizierte Werkzeuge komplett weg. Außerdem werden die Informationen die im Feld auf dem mobilen Klienten angezeigt werden derart reduziert, dass lediglich aussagekräftige Werte, die damit Entscheidungen unterstützen können, angezeigt werden. In dem vorherigem Kapitel habe ich den Problembereich beschrieben, nun möchte ich die verschiedenen Prozesse skizzieren die ein Nutzer durchführen könnte.

\begin{figure}[H]
	\centering
 	 \includegraphics[scale=0.2]{graphics/bpmn_business-processes.jpg} 
	\caption{BPMN (Business Process Model and Notation) Modell relevanter Aktionen welche in dem System durchgeführt werden, F. H. Euteneuer 2013}
	 \label{fig:model_business-processes}
\end{figure}

Ich habe drei verschiedene Hauptaktionen identifiziert, die ein Nutzer durchführen könnte: Das manuelle Messen von Werten, das manuelle Editieren bereits gemessener Werte, und das automatische kontinuierliche Messen. Die Abbildung \ref{fig:model_business-processes} veranschaulicht mittels eines \gls{UML} Activity Diagrams (\gls{UML} Aktivitäten Diagramm) die einzelnen Abläufe dieser Aktionen.

Die manuelle Messung beginnt mir der normalen Messung der Werte. Im zweiten Schritt erfolgt die Eingabe der Werte in das System. Die Werte werden automatisch auf ihre Validität hin überprüft, und erste einfache statistische Analysen werden erstellt. Diese erste Statistik ist erforderlich um Informationen über die Qualität der Messung zu erhalten, und dem System die Möglichkeit zu bieten fehlerhafte Messungen zu bemängeln und Neumessungen vorzuschlagen.

Der Nutzer wird die Möglichkeit haben vergangene Messungen manuell zu bearbeiten. Dazu muss ein Datensatz (Üblicherweise ein Messwert) ausgewählt werden, und der Nutzer kann dann entscheiden ob die betreffende Messung wiederholt werden soll, oder die Werte manuell geändert werden sollen. Bei einer Wiederholung der Messung wird die Prozesskette der manuellen Messung durchlaufen.

Die automatische Messung ist die wichtigste Funktion des Systems, und stellt eine der Innovationen dar. Obwohl das Verfahren ein anderes ist, sind die zu Grunde liegenden Funktionen der manuellen und der automatischen Messung sehr ähnlich. Als initiale Handlug muss das Sensor Netzwerk eingerichtet werden, dazu gehören die Eingabe der Metadaten, wie Sensor-Typ und -Position oder benutztes geographisches Referenz System. Welche Parameter tatsächlich benötigt werden um ein lauffähiges Sensoren Netzwerk einzurichten, wird in später folgenden methodischen Teilen genauer beschrieben. Nach der Einrichtung des Netzwerkes hat der Nutzer die Möglichkeit kontinuierliche automatische Messungen zu starten, und später auch wieder zu stoppen. Auch ein nachträgliches Ändern der eingegebenen Parameter ist möglich.

Zum Abschluss dieses Kapitels möchte ich nochmal darauf hinweisen, dass die hier beschriebene Liste an Funktionalitäten lediglich ein sehr rudimentäres System beschreiben, und demnach auch keinen Anspruch auf Vollständigkeit erhebt.


\section{Produktfunktionen}
Was muss das System nun an Funktionalität anbieten, um den genannten Ansprüchen gerecht zu werden? Um diese Frage besser beantworten zu können möchte ich einige Anwendungsfälle beschreiben, in denen verschiedene Nutzergruppen für sie interessante Aufgaben mit Hilfe des System lösen werden. Nach einer kurzen Beschreibung der verschiedenen Nutzergruppen werde ich jeden Anwendungsfall in tabellarischer Form analysieren.

\begin{figure}[H]
	\centering
 	 \includegraphics[scale=0.6]{graphics/uml_functionalities.jpg} 
	\caption{UML Anwendungsfalldiagram des beschriebenen Systems und der einzelnen Nutzergruppen mit ihren Anwendungsfällen, F. H. Euteneuer 2013}
	 \label{fig:model_functionalities}
\end{figure}

\subsection{Nutzergruppen}
Die Analyse der verschiedenen Nutzergruppen des Systems bildet eine wichtige Informationsquelle für die Konzeption des Systems. Anstatt wahllos Funktionen zu beschreiben, die vermutlich in ein System gehören sollten, werden so imaginäre aber präzise definierte Nutzer identifiziert und befragt, was sie mit diesem System anfangen wollen. Die Abbildung \ref{fig:model_functionalities} enthält bereits die zwei identifizierten wichtigsten Nutzergruppen die in die Prozesse involviert sind.

\subsubsection{Feldingenieure}
Die Gruppe der Feldingenieure kann als die Gruppe der Ausführenden Personen beschreiben werden, die im direkten Kontakt zu den Sensoren stehen oder selbst manuell die Messungen durchführen. Für die Kommunikation mit dem System verwenden sie einen mobilen Klienten, dadurch sind sie technisch bestimmten Restriktionen unterworfen. In der Folgenden Liste sind die wichtigsten benannt:
\begin{itemize}
\item Kleine Anzeigefläche auf dem Mobilen Klienten (Qualität der Visualisierung ist limitiert)
\item Fehlende oder mangelhafte Eingabemöglichkeiten (z.B.: Eingabe nur durch virtuelle Tastatur auf einem Mobilen Computer)
\item Hohes Gewicht von nicht mobilen Klienten (z.B.: Verwendung eines konventionellen Notebooks als mobile Lösung, für Arbeiten im stehen oder Laufen aber zu schwer)
\end{itemize}
Nichtsdestotrotz definiert diese Nutzergruppe die herausfordernsten Anforderungen an das System, beispielsweise durch die Implementierung einer intelligenten Möglichkeit zur Visualisierung. Da diese Nutzergruppe die eigentliche Zielgruppe des Systems darstellt, sollten die Anforderungen dieser Gruppe zu gut als möglich erfüllt werden.

\subsubsection{Bürokraft}
Normalerweise sind die Nutzer Feld- und Büroingenieur vereinigt in einer Person. Projekte die sich mit der Überwachung von Bauwerken befassen können in zwei Teile untergliedert werden. Ein Teil ist für die Installation des Netzwerkes, für etwaige manuelle Messungen und für die Betreuung bestehender Sensornetzwerke zuständig, während sich der andere Teil um die Auswertung der eingehenden Daten, die \"Postprozessierung\" (DE: Nachbearbeitung) und die Interpretation der Daten kümmert. Durch die meist sehr komfortabel ausgestattete Informationstechnologie in den Büros erwarte ich hier niedrigere technische Anforderungen an das System.

\subsection{Anwendungsfälle}
Die Abbildung \ref{fig:model_functionalities} zeigt die grundlegenden unterschiedlichen Anwendungsfälle der zwei im vorherigen Kapitel beschriebenen Nutzergruppen. Anwendungsfälle die in der Abbildung genannt werden repräsentieren Handlungsfolgen des jeweiligen Nutzers, die innerhalb des Anwendungsfalles abgeschlossen sein müssen, also ein festgelegtes Ziel erreichen müssen. Ich werde nun die Anwendungsfälle wie bereits erwähnt in tabellarischer Form mit weiteren Parametern beschreiben. Dieser Arbeitsschritt ist essentiell für Planung und Konzeption eines Systems, da hierdurch der tatsächliche Bedarf der Nutzer und damit die zu implementierenden Funktionalitäten beschrieben werden.

Ich werde alle Anwendungsfälle zunächst mit einem Text einleiten, und dann in der Tabelle die wichtigsten Eigenschaften beschreiben. Dazu gehören das festgelegte Ziel eines Anwendungsfalles oder die sogenannte Nachbedingung, die beschreibt in welcher Situation sich das System nach dem erfolgreichem Durchlaufen eines Anwendungsfalles befindet. Anschließend werden die einzelnen zu durchlaufenden Schritte der Anwendungsfälle in einem \gls{UML} Aktivitäten Diagramm veranschaulicht.

Ich habe die verschiedenen Anwendungsfälle in drei Gruppen gegliedert. Jeweils zwei Anwendungsfälle decken die Gebiete des Daten Managements, der Daten Analyse und der eigentlichen Messung ab.

Die hier vorgestellten Anwendungsfälle repräsentieren keinesfalls alle Arbeitsabläufe die möglich sind, sondern sollen nur einen möglichen Lösungsweg beschreiben, der in dem Prototyp implementiert werden könnte.

\subsubsection{Management}
Management soll hier sowohl für das Management von Daten als auch für das Einrichten und die Betreuung des Systems stehen. Erster Anwendungsfall soll das Aufsetzen eines Sensornetzwerkes beschreiben. Das kann auch als initiale Handlung bei der Verwendung des Systems gesehen werden, und ist damit eine Art Vorbedingung für alle nachfolgenden Anwendungsfälle. Die Tabelle \ref{table:use case description of "Set up network"} beinhaltet die zentralen Eigenschaften dieses Anwendungsfalles.

Das Datenmanagement ist ein erforderlicher Teil eines Systems das sich mit Daten und Metadaten befasst. Daten zu sammeln ohne sie nutzen zu können würde keinen Sinn ergeben, somit ist ein Export der Daten aus dem System heraus eine obligatorische Funktion des Systems. Dieser Anwendungsfall kann also auch als finale Handlung gesehen werden, die unter Verwendung des Systems durchgeführt werden wird. Die zweite Tabelle \ref{table:use case description of "Export data"} beinhaltet detailierte Informationen über diesen "Datenexport" genannten Anwendungsfall.

\begin{table}[H]
\centering
\begin{tabular}{l | p{11cm}}
Name & Einrichtung des Netzwerkes\\ \hline 
Nutzergruppe & Feldingenieur\\ \hline 
Ziel & Eingabe aller Metadaten über die verbundenen Sensoren und Initialisierung des Netzwerkes\\ \hline 
Vorbedingung & Das Netzwerk existiert, ist eingerichtet und ist mit dem System verbunden\\ \hline 
Nachbedingung & Funktionierendes Netzwerk mit allen Sensoren\\ 
\end{tabular}
\caption{Tabellarisierte Beschreibung aller Charakteristika des Anwendungsfalls "Einrichtung des Netzwerkes"} 
\label{table:use case description of "Set up network"}
\end{table}

\begin{table}[H]
\centering
\begin{tabular}{l | p{11cm}}
Name & Datenexport\\ \hline 
Nutzergruppe & Bürokraft\\ \hline 
Ziel & Auswahl der Daten und Export in einem bestimmten Format\\ \hline 
Vorbedingung & Auswahl der Daten nach bestimmten Parametern und spezifiziertes Exportformat\\ \hline 
Nachbedingung & Ausgewählte Daten liegen vollständig physikalisch im definiertem Format vor\\ 
\end{tabular}
\caption{Tabellarisierte Beschreibung aller Charakteristika des Anwendungsfa'
lls "Datenexport"}\label{table:use case description of "Export data"}
\end{table}

Die Abbildung \ref{fig:bpmn_use-case_management} zeigt die Anwendungsfälle die sich mit der Thematik des Managements befassen in einem \gls{UML} Aktivitätsdiagramm. Die beiden Abläufe weisen keine Interaktionen untereinander auf und sind damit vollständig unabhängig voneinander. Chronologisch hingegen sollte das Einrichten des Netzwerkes vor dem Datenexport erfolgen.

Der Ablauf des Einrichtens des Netzwerkes beinhaltet zwei wichtige Aktionen: Zum Einen das Einrichten der Datenbank und zum Anderen die Eingabe der Sensorparameter. Dies sind die zwei zentralen teile dieses Anwendungsfalls, und ein scheitern nur eines dieser Aktionen würde zu einem nicht funktionierendem Netzwerk führen. Die Bearbeitung der "Netzwerkeinstellungen", also der einzelnen eingegebenen Parameter, führt zu einem erneutem Durchlaufen der gesamten Prozesskette. Damit sollen fehlerhafte Eingaben vermieden werden, diese Funktion stellt eine Art Assistenzsystem dar, das durch alle wichtigen Einstellmöglichkeiten führt.

Der Anwendungsfall "Datenexport" ist um Einiges einfacher als der vorherige, im Prinzip ähnelt er den meisten klassischen Exportfunktionen oder Speicherfunktionen. Einzig die Auswahl der zu exportierenden Daten durch das setzen eines Zeitrahmens stellt eine größere Herausforderung an das System dar.

\begin{figure}[H]
	\centering
 	 \includegraphics[scale=0.24]{graphics/bpmn_use-cases_management.jpg} 
	\caption{BPMN (Business Process Model and Notation) Modell der Management Anwendungsfälle, F. H. Euteneuer 2013}
	 \label{fig:bpmn_use-case_management}
\end{figure}


\subsubsection{Messungen}
Der Teil des Systems der sich mit den eigentlichen Messungen beschäftigt, könnte als der wichtigste Teil gesehen werden, stellt aber an sich keine wesentliche Innovation dar. Dieser Teil beschreibt als Einziger die manuelle Bearbeitung oder Veränderung der Daten in der Datenbank.

In diesem Teil habe ich zwei wichtige Anwendungsfälle identifiziert: Der Erste beschäftigt sich mit dem initialen Eingabe von Daten die durch die Messungen produziert wurden. Die Charakteristiken dieses Anwendungsfalles sind in der Tabelle \ref{table:use case description of "Measure data"} beschrieben. Im Gegensatz zu den automatischen Messungen beschreibt dieser Anwendungsfall die manuelle Eingabe von nur einem Datensatz.

Der Zweite Anwendungsfall behandelt das manuelle Bearbeiten der bereits in der Datenbank gespeicherten Werte. Die Tabelle \ref{table:use case description of "Edit data"} beinhaltet alle wichtigen Informationen dazu. Das Bearbeiten von Daten gehört zu den Standardoperationen für System die sich auf Datenbanken stützen. Dennoch ist es wichtig zu beschreiben, wie dieser Anwendungsfall mit dem der Messungen zusammenhängt. Im Falle von Neumessungen bestimmter Werte oder des Validieren von Daten wird die Prozesskette der Messung durchlaufen obwohl es im Grunde eine Bearbeitung bereits bestehender Werte ist.

\begin{table}[H]
\centering
\begin{tabular}{l | p{11cm}}
Name & Messung\\ \hline 
Nutzergruppe & Feldingenieur\\ \hline 
Ziel & Eingabe aller Messergebnisse von Einzelmessungen per Hand\\ \hline 
Vorbedingung & Lauffähiges System mit Verbindung zur Datenbank\\ \hline 
Nachbedingung & Gültige Daten in der Datenbank mit vollständigen Metadaten\\ 
\end{tabular}
\caption{Tabellarisierte Beschreibung aller Charakteristika des Anwendungsfalls "Messung"} 
\label{table:use case description of "Measure data"}
\end{table}

\begin{table}[H]
\centering
\begin{tabular}{l | p{11cm}}
Name & Datenbearbeitung\\ \hline 
Nutzergruppe & Bürokraft\\ \hline 
Ziel & Auswahl der Datensätze nach Parametern und Bearbeitung der Werte per Hand oder durch Neumessung\\ \hline 
Vorbedingung & Lauffähiges System mit Verbindung zur Datenbank\\ \hline 
Nachbedingung & Veränderte Daten in der Datenbank mit vollständigen Metadaten\\ 
\end{tabular}
\caption{Tabellarisierte Beschreibung aller Charakteristika des Anwendungsfalls "Datenbearbeitung"} 
\label{table:use case description of "Edit data"}
\end{table}

Bei manuellen Messungen sind die Aktionen die in der oberen Reihe des Aktivitätendiagrammes \ref{fig:bpmn_use-case_measuring} angegeben werden unabdingbar. Das System wird nach der Durchführung der Messungen eine schnelle Analyse der Messergebnisse durchführen um deren Qualität zu bewerten. Nach diesem Schritt wird das System entweder auf mögliche Fehler in der Messung hinweisen, oder die Messwerte direkt in die Datenbank schreiben. 

Die zweite Linie des Diagramms beschreibt die Handlungskette der Datenbearbeitung. Der Nutzer hat zwei Möglichkeiten Daten nachträglich zu bearbeiten, zum Einen durch die erneute Messung der Daten, zum Anderen durch das manuelle Bearbeiten, also die Eingabe neuer Werte und das Überschreiben der alten Werte. 

\begin{figure}[H]
	\centering
 	 \includegraphics[scale=0.24]{graphics/bpmn_use-cases_measurement.jpg} 
	\caption{BPMN (Business Process Model and Notation) Modell der Anwendungsfälle des Teiles Messungen, F. H. Euteneuer 2013}
	 \label{fig:bpmn_use-case_measuring}
\end{figure}

\subsubsection{Analyse}
In den vorherigen Kapiteln habe ich oft von einer Analyse der Daten gesprochen, bislang aber nichts Genaueres dazu gesagt. Die Analyse stellt den wohl schwierigsten Teil der Konzeption dar, weil dieser die eigentliche Innovation darstellt. Die Analyse die ich hier beschreiben werde unterscheidet sich allerdings von der auch oft erwähnten Überprüfungen der Daten. Diese "ad hoc" Statistiken stellen lediglich eine schnelle und unpräzise Vorabinformation über die Qualität der Daten dar. Um die Frage beantworten zu können ob Messungen einen Sinn ergeben, also die Konsistenz der Daten beurteilen zu können, bedarf es komplexere Algorithmen, die außer den eigentlich zu beurteilenden Daten auch die vorhergegangenen Beobachtungen (Epochen) mit einbeziehen. Diese Algorithmen erzeugen trotzdem nur eine schnelle Übersicht über Mögliche Abweichungen oder Ereignisse, nichtsdestotrotz wird der Nutzer mithilfe dieser Analysen in der Lage sein Mögliche Verfahrensfehler bei den Messungen direkt im Feld zu identifizieren. Die Funktionalität wird in der tabellarischen Beschreibung \ref{table:use case description of "Analyse data"} des Anwendungsfalls "Datenanalyse" charakterisiert.

Der zweite Teil des Analyse Bereiches basiert auf einer visuellen Analyse der Daten durch den Nutzer. Kern dieser Funktionalität sind eine Visualisierung des zu überwachenden Objektes und der Ergebnisse bereits erfolgter Analysen und werden in der Tabelle \ref{table:use case description "Visualise data"} charakterisiert. Das System wird eine grafisches Modell des zu überwachenden Objektes generieren, und die Optionen anbieten Messergebnisse und berechnete Statistiken ebenfalls in die Grafik zu integrieren. Der Nutzer ist dann in der Lage Messungen in Echtzeit zu überwachen, und beispielsweise auch Auswirkungen von Experimenten zu verfolgen. Die Art und Weise der Visualisierung von Objekt, Messung und Statistik wird genauer im methodischen Teil der Arbeit beschrieben.

\begin{table}[H]
\centering
\begin{tabular}{l | p{11cm}}
Name & Datenvisualisierung\\ \hline 
Nutzergruppe & Feldingenieur\\ \hline 
Ziel & Visuelle Unterstützung bei Messungen und Interpretation\\ \hline 
Vorbedingung & Vorhandene Metadaten für die Betreffende Messung (z.B. verwendeter Sensor, Orientierung, ..)\\ \hline 
Nachbedingung & Aussagekräftige und "unterstützende" Grafik der ausgewählten Daten\\
\end{tabular}
\caption{Tabellarisierte Beschreibung aller Charakteristika des Anwendungsfalls "Datenvisualisierung"} 
\label{table:use case description "Visualise data"}
\end{table}

\begin{table}[H]
\centering
\begin{tabular}{l | p{11cm}}
Name & Datenanalyse\\ \hline 
Nutzergruppe & Feldingenieur\\ \hline 
Ziel & Erhalt von Informationen über Gültigkeit der Daten verglichen mit vergangenen Epochen\\ \hline 
Vorbedingung & Bereits mehr als zwei Messungen liegen vor\\ \hline 
Nachbedingung & Informationen über Gültigkeit der Messergebnisse\\ 
\end{tabular}
\caption{Tabellarisierte Beschreibung aller Charakteristika des Anwendungsfalls "Datenanalyse\\"} 
\label{table:use case description of "Analyse data"}
\end{table}

Die Abbildung \ref{fig:bpmn_use-case_analysis} zeigt die Ereignisabläufe der beiden ausgewählten und beschriebenen Anwendungsfälle des Analyseteils des Systems. In der ersten Reihe ist als wichtiger Schritt die Definition der historischen Daten hervorzuheben. Erst damit werden Werte aussagekräftig.

Die Visualisierung der Daten ins untergliedert in zwei Typen. Der Nutzer kann zwischen der Visualisierung der gemessenen Werte und der berechneten Statistiken wählen. Für die Visualisierung der Statistiken müssen die Statistiken bereits berechnet worden sein. Die Durchführung der Analyse kann auch aus der Visualisierungs-Funktion heraus aufgerufen werden.

\begin{figure}[H]
	\centering
 	 \includegraphics[scale=0.24]{graphics/bpmn_use-cases_analysis.jpg} 
	\caption{BPMN (Business Process Model and Notation) Modell der Anwendungsfälle des Teiles Analyse, F. H. Euteneuer 2013}
	 \label{fig:bpmn_use-case_analysis}
\end{figure}


\section{Nicht-funktionale Anforderungen}
In diesem Kapitel beschreibe ich die nicht-funktionalen Anforderungen an das System beschreiben. Nicht-funktional bedeutet, dass beschrieben wird wie das System wo eingesetzt werden soll. Es handelt sich also um Eigenschaften des Systems, die keine technischen Funktionen sind.

\begin{description}
\item[Portabilität] Da das System besonders auf die Kommunikation zwischen der Arbeit im Feld und der Auswertung im Büro zugeschnitten ist, werden besondere Anforderungen an die Möglichkeit der Portierung auf verschiedene Betriebssysteme der mobilen Klienten gestellt. Das System soll damit Plattform-unabhängig sein. 
\item[Rechenleistung] Wie im Punkt zuvor beschrieben soll das System auf unterschiedlichen mobilen Klienten eingesetzt werden können. Deren Auslegung auf die Mobilität geht meist zu Lasten der Rechenleistung der Hardware. Damit das Arbeiten mit dem System dennoch effizient und einfach gehalten werden kann, muss der mobile Teil des Systems für diese kleine Rechenleistung zugeschnitten werden.
\item[Einfachheit] Die Hauptnutzer des Systems werden Ingenieure sein, die oft unter schwierigen Bedingungen arbeiten müssen. Die Bedienung eines IT-Systems sollte demnach so intuitiv und einfach wie möglich gestaltet werden, damit der Nutzer nicht unnötig Zeit in die Suche von Funktionen oder das Verstehen von Grafiken aufbringen muss.
\end{description}

Diese Liste beschreibt sehr grundlegende Anforderungen an ein System. Vor Allem wenn es sich um ein System handelt, das eine mobile Komponente beinhaltet. Dennoch sind dies sehr zentrale Anforderungen die in den nachfolgenden Kapiteln der Arbeit detaillierter aufgegriffen werden. Bei der näheren Beschreibung der Komponenten und ihrer Funktionsweise wird überprüft inwieweit diese den nicht-funktionalen Anforderungen gerecht wird. Ein System dass funktioniert, aber nicht in der geplanten Umgebung eingesetzt werden kann, erfüllt genauso wenig die Anforderungen, wie ein System dass den funktionalen Anforderungen nicht entspricht.


\chapter{Prototyp}
Ein Prototyp stellt in diesem Fall eine Software dar, die demonstrieren soll wie die Architektur und ein Teil der geplanten und beschriebenen Funktionen aussehen könnten. Mit Hilfe dieses Prototypen kann auch leicht evaluiert werden inwieweit die Funktionen die Probleme lösen können, und ob eine Weiterentwicklung in der eingeschlagenen Richtung Sinn machen würde. Üblicherweise stellt ein Prototyp lediglich einen Ausschnitt des Gesamtkonzeptes dar, im Folgenden möchte ich festlegen worauf ich mich mich verstärkt konzentrieren werde, und was in dem Prototypen enthalten sein wird. Die Grundlage für meine Auswahl waren zum Einen die Realisierbarkeit einer Implementierung in der kürze der Zeit und zum Anderen die mir zur Verfügung stehenden Mittel.

In der Grafik \ref{fig:model_domain} habe ich die einzelnen Teile des Systems beschrieben. Für das Komplettsystem sind zweifelsohne alle Teile wichtig. Die Verbindung aus Server, mobiler Klient und automatische Sensoren stellt jedoch den Teil mit der größten Innovation dar. Ohne die beiden anderen Teile vollständig außer Acht zu lassen werde ich mich in den nachfolgenden Kapiteln verstärkt mit dieser Verbindung und der Beschreibung der drei Teile befassen.

Die Architektur und Funktionsweise des Prototypen wird im dritten Teil der Arbeit beschrieben. Dort werden auch die in diesem Teil beschriebenen Arbeitsabläufe getestet und evaluiert. Am Ende soll eine Aussage darüber möglich sein, ob das System funktionstüchtig konzipiert wurde, ob alle wesentliche Teile enthalten sind, und ob für die Nutzergruppen ein Mehrwert geschaffen wurde.

\subsection{Fundamentals of Photovoltaik systems}

A Photovoltaic systems transduces solar radiation into electricity. The efficiency of a system depends on the properties of the used photovoltaic cells. Wagner (2010) \citen{Wagner2010} \\

\begin{figure}[hbt!]
\centering
\includegraphics[width=0.7\textwidth]{phase2/group2/figure/pv-system-configuration.png}
\caption{Architecture of Photovaltaic system}
\label{fig:PV_system}
\end{figure}

The nominal power of a cell is given in Watt and often denoted as \(W_p\) (Watt Peak). It is acquired under standard test condition (STC an international standard) from the manufacturer of the photovoltaic cell. STC means, that the cell temperatur is \(25^\circ C\) and the irradiance is 1000 \(W/m^2\). The nominal power can be used to compare the power of photovoltiac cells of different manufactures. 
\\
The efficiency is given by the ratio between the produced energy and the energy radiated to the cell. It depends on the temperature of the photovoltaic cells. With increasing temperature the efficiency decreases. Often only one efficiency coefficient, including the efficiency of the power inverter, cable and accumulator, is given by the manufacturers.
\\
In addition a performance ratio is given most times. It is the ratio of the actual and the desired gain of the photovoltaic cell. It depends not only on the cell itself but also on the weather conditions, respectivly on the location. The desired gain is the efficiency of the installation under STC, assuming that the efficiency of the power inverter is \(100\%\). For a thin-layerd slicon cell, which is most common, the Performance ratio (PR) is \(84\% \) for Germany.
Wagner (2010)


\subsection{Fundamentals of Solar thermal systems}

Solar thermal collectors are devices, which transform radiation emitted by sun into heat. Basically, a collector consists of a box with a black glazed cover, containing a pipe, which uses all the space in the box. The radiation is adsorbed by the black glazing and heats up a fluid, such as water or oil, which runs through the pipe equipped in the  collector. The heated fluid can then be stored in insulated tanks, if it cannot be used directly.    In general solar collectors are used for use water in a household or for heating.\citen{Kalogirou}

basically, the efficiency of a solar thermal collector depends on the absorption of the glazing, the input fluid temperature , the average air temperature as well as the solar radiation of the region.
% global irradiance
\subsection{Global Irradiance}

The most important input value to calculate the Solar potential is the global radiation. The global radiation comprises the direct solar radiation and the diffuse radiation resulting from reflected or scattered sunlight. It depends on the location, the orientation of the roof surface and the inclination of the roof surface. The location is the latitude \(52^\circ 31' 45.0''\) and longitude \(13^\circ 19' 42.6''\)  of Moabit in Berlin. The orientation and the inclination are calculated from the CityGML Lod2 geometry. (European Database of Daylight and Solar Radiation)

\subsubsection{Orientation of the roof}

The azimuth angle \(\alpha\) represents the orientation of the roof. It is given by the angle between the normal vector of the roof \(n_r\) surface and the normal vector of xz plane \(n_{xz}\). 
\begin{eqnarray}
\label{eq:angle2vec}
\alpha = cos^{-1} ( \frac{\vec{n_r} \vec{n_{xz}}}{|n_r|  |n_{xz} | })
\end{eqnarray}

The direction of the normal vector is defined by the order of the point sequence forming the polygon ring of the roof surface. To calculate the orientation the positive normal vector is needed. If \(n_r\) has a negative z component \(n_r\) is converted to the positive normal vector. The calculated angle \(\alpha\) is always in a range between \(0^\circ-180^\circ\). Because the azimuth angle has a range of \(0^\circ-360^\circ\) we have to consider the sign of the x component of \(n_r\). Is the x component negative the azimuth angle is calculate with
\begin{eqnarray}
\alpha_o = 360^\circ - \alpha
\end{eqnarray}
else the azimuth angle \(\alpha_o\) is equal to \(\alpha\). 

\subsubsection{Inclination of the roof}
The inclination of the roof influences the input of solar radiation strongly and has to be considered. It is calculated as the angle between the normal vector of the roof surface \(n_r\) and the normal vector of the xy plane \(n_{xy}\). To calculate this angle equation \ref{eq:angle2vec} is used. Again the calculated angle ranges between \(0^\circ-180^\circ\), although the maximal tilt angle is \(90^\circ\). Therefore the tilt angle is
\begin{eqnarray}
t = 180^\circ - \alpha 
\end{eqnarray}
if \(\alpha > 90^\circ\) else \(t\) is equal \(\alpha\). For further calculation all roof surfaces with an inclination \(t<8^\circ\) are considered to be flat roofs and the angle \(t\) is set to \(0^\circ\).

\subsubsection{Satel-Light Database}
The estimation of the global irradiation for flat roof surfaces is simple, because no diffuse radiation effects the global radiation. For inclined surfaces the diffuse radiation resulting from reflected or scattered sunlight has to be considered. Because this is very complex we decided to use the European Database of Daylight and Solar Radiation (Satel-Light), which provides for every location within Europe the global radiation on tilted else well as flat roof surfaces with arbitrary orientation. We created a LookUp table for Berlin, which contains the global radiation for the inclination in \(10^\circ\) degree steps and the orientation in \(45^\circ\) steps. The resulting table is shown in figure \ref{fig:rad_table}.

\begin{figure}[hbt!]
\centering
\includegraphics[width=0.9\textwidth]{phase2/group2/figure/table_global_radiation.png}
\caption{LookUp table for the global radiation for Berlin}
\label{fig:rad_table}
\end{figure}

\subsection{Reduction of Roof Surface Area due to Roof Equipment}
\label{sec:roofarea}
The main problem when using a LOD 2 model is that the suitable roof area is not known. Fact is, that not 100\% of the roof can be used to install solar panels, since roofs may have dormers, antennas or chimneys which are not part of the LOD2 model. But the roof equipment is considered for the Solar Atlas Berlin and with use of this data source empirical reduction factors may be computed. Therefore group 2 implemented a program, which reads a cityGML file, containing all buildings of the test area (~1000 buildings) and computes the reduction factor as in Equation \ref{eq:roof}.

\begin{align}
\label{eq:roof}
r_E &= \frac{A_{SAB}}{A_{LOD2}} \\\notag\\\notag
\text{with:}&\\\notag
r_E &:\text{ empirical reduction factor}\\\notag
A_{SAB} &: \text{Roof Area according to solar atlas Berlin }\\\notag
A_{LOD2} &: \text{ roof area calculated from LOD2 model in 3D CityDB }\\\notag
\end{align}

The Solar Atlas Berlin provides different areas for photovoltaic and solar thermal systems. Therefore also different reduction factors are computed. The reduction factors are also separated between exclusively flat roof and mixed roofs, contain flat as well as tilted surfaces. This is important because flat roofs are more likely equipped. Additionally, the age class of the building has been taken into account. Tables \ref{tab:roofarea_st} and \ref{tab:roofarea_pv} show the result, including the number of surface and the variance of the value. It can be seen that, some reduction factors are not representative, because not enough roofs of this type are in the test area. With the use of a database of entire Berlin might fix the Problem.

 

\begin{table}
\centering 

\begin{tabular}{|c||c|c|c||c|c|c|}
  \hline
  \multirow{2}{*}{Age Class} & \multicolumn{3}{|c||}{flat roof} &  \multicolumn{3}{|c|}{mixed roof}\\
  \cline{2-7}
  & $r_E$ & count & $\sigma^2$  & $r_E$ & count & $\sigma^2$\\
  \hline
1899&0.198&94&0.016&0.440&199&0.056\\\hline
1918&0.152&56&0.019&0.435&233&0.059\\\hline
1932&0.167&13&0.017&0.506&9&0.048\\\hline
1945&0.304&2&0.0008&0.765&1&0.000\\\hline
1961&0.221&74&0.013&0.581&56&0.049\\\hline
1974&0.198&90&0.015&0.371&17&0.080\\\hline
1993&0.219&66&0.006&0.293&5&0.016\\\hline
2012&0.149&91&0.019&0.487&21&0.165\\\hline
\end{tabular}
\caption{empirical reduction factors for calculation of energy gain using solar thermal collectors }
 \label{tab:roofarea_st}
\end{table}

\begin{table}
\centering 
\begin{tabular}{|c||c|c|c||c|c|c|}
  \hline
  \multirow{2}{*}{Age Class} & \multicolumn{3}{|c||}{flat roof} &  \multicolumn{3}{|c|}{mixed roof}\\
  \cline{2-7}
  & $r_E$ & count & $\sigma^2$  & $r_E$ & count & $\sigma^2$\\
  \hline
1899&0.127&94&0.019&0.349&199&0.050\\\hline
1918&0.097&56&0.016&0.323&233&0.053\\\hline
1932&0.056&13&0.008&0.403&9&0.076\\\hline
1945&0.000&2&0.000&0.730&1&0.000\\\hline
1961&0.147&74&0.015&0.498&56&0.050\\\hline
1974&0.129&90&0.016&0.289&17&0.081\\\hline
1993&0.128&66&0.011&0.146&5&0.013\\\hline
2012&0.090&91&0.016&0.399&21&0.147\\\hline
\end{tabular}
\caption{empirical reduction factors for calculation of energy gain using photovoltaic modules }
 \label{tab:roofarea_pv}
\end{table}


\subsection{Shadowing}
\label{sec:shadowing}
Roof surfaces may be shadowed by several object, such as other buildings, trees or even equipment on the roof itself. Since, the data source is a LOD 2 model only other buildings can be considered, because trees and roof equipment are not part of the model. The consideration of the shadowing may be done with a complicated illumination computation. Since this project is limited in time this is not possible. Therefore only a simple approach is applied, which completely ignore buildings which are shadowed, rather then adjusting the daily global irradiance on the specific surface.
Within the simple approach a building is neglected, if there is a neighboured building which is $x$ higher and is within a certain radius $r$. Only buildings between an azimuth of 90° to 270° are taken into account, as shown in Figure \ref{fig:shadow}.

\begin{figure}[ht]
	\centering
	\includegraphics[width=0.7\textwidth]{phase2/group2/figure/fig_shadow.png}
	\caption{The green buildings are candidates, which may shadow the blue building}
	\label{fig:shadow}
\end{figure}

The implemented algorithm starts with iterating over all buildings and storing them in a spatial tree to make them easy queryable. After that before a potential of a building is calculated it will be checked if there are candidate buildings which are within the radius $r$. The list of candidates is checked for buildings which also meet the other conditions, such as an azimuth between $90^\circ$ and $270^\circ$ and if the building is $x$ times higher. If both conditions are met, the building will be neglected for the potential calculation. 
\subsection{Calculation of the potential}

The calculation begins with the reduction of the roof area according to the building age class as described in chapter \ref{sec:roofarea}. If the roof surface is a flat roof ($tilt < 8 ^\circ$) the modules cannot directly mounted of the roof surface. To bring the modules in an optimal tilt angle they are mounted on a mounting system of the roof. Because they are tilted they might shadow the neighboring modules, therefore a certain distance between the modules is necessary as shown in Figure \ref{fig:flatroof}. The distance has to be $c=2.75$ times longer then the width of the ground of the module $w$.
%TODO: ADD REFERENCE!
\begin{figure}[ht]
	\centering
	\includegraphics[width=0.5\textwidth]{phase2/group2/figure/flatroofreduction.png}
	\caption{Solar Modules mounted on a mounting system of a flat roof}
	\label{fig:flatroof}
\end{figure}
Furthermore the global irradiance has to be picked from the look-up table. According to Solar Atlas Berlin \citen{solaratlas} areas with a global irradiance less than $905 Wh/m^2$ have to be neglected. Also surface are neglected which have a smaller area then $5m^2$. It is assumed that it is economically not worth the mount modules on such a small roof. The actual calculation of the potential highly depends on the type of module. This will be explained in the following sub sections.



\subsubsection{Potential of Photovoltiac systems}
To calculate the potential of photovoltiac systems two approaches were applied. First the estimation as described by Wagner (2010) \citen{Wagner2010} was implemented. According to Wagner (2010) the energy gain of a photovoltiac system can be calculated with 

\begin{align}
\label{eq:pv_calc}
E &= M \cdot GA \cdot \frac{P}{E_0} \cdot PR \cdot \eta_{EUR} \cdot \eta_l \\\notag\\\notag
\text{with:}&\\\notag
E &:\text{total energy gain per year \(kWh/a\)} \\\notag
E_0 &:\text{1000 \(W/m^2\)} \\\notag
M &:\text{Number of Modules} \\\notag
PR &:\text {Performance ratio} \\\notag
P &:\text{nominal power \(W\)} \\\notag
\eta_{EUR} &:\text{euro inverter efficiency} \\\notag
\eta_l&:\text{transmission efficiency} \\\notag
GA &:\text{Global Irradiation \(kWh/m^2 a\)} \notag
\end{align}

The parameters \(P\),\(\eta_{EUR}\),\(\eta_l\) and \(M\)depend on the photovoltaic cell and the inverter. Values for these parameters are taken from real photovoltiac cells. For the calculations the silicon cell BP 585F from BP Solar \citen{BPSolar} combined with the inverter SP 2500-450 from the company Sun Power \citen{SunPower} has been used. The inverter efficiency is \(\eta_{EUR} = 15 \%\) and the transmission efficiency is set to \(\eta_l = 9\%\). The nominal power of the cell is \(P_0 = 85 W\). This calculation method allows to use real data of photovoltiac cells and considers the inverter.
\\

Because the the Solar Atlas Berlin is the only available reference, finally a second approach according to the Solar Atlas was used. The calculation of the photovoltiac energy is simplified and finally done with equation \ref{eq:pv_calc_SAB}. Where the efficiency coefficient is set to \(e=15\%\) and the system area is the reduced roof surface area.

\begin{align}
\label{eq:pv_calc_SAB}
E &= A \cdot GA \cdot PR \cdot e \\\notag\\\notag
\text{with:}&\\\notag
E &:\text{total energy gain per year \(kWh/a\)} \\\notag
A &:\text{System Area \(m^2\)} \\\notag
e &:\text{efficiency coefficient} \\\notag
GA &:\text{Global Irradiation \(kWh/m^2 a\)} \notag
\end{align}





\subsubsection{Solar Thermal}
The calculation of the potential of solar thermal modules was done as described in Struckmann (2008) \citen{struckmann2008}. According to Struckmann (2008) the useful energy gain $Q_U$ of a solar thermal module is calculated with the formula shown in Equation \ref{eq:st_calc}. Figure \ref{fig:st_module} shows a sketch of a typical solar thermal module and shows the parameter, which are necessary to compute $Q_U$  

\begin{align}
\label{eq:st_calc}
Q_U &= F_R  A \left( I \tau \alpha - U_L \left(T_i - T_a \right) \right)\\\notag\\\notag
\text{with:}&\\\notag
F_R &:\text{ Efficiency Coefficient of the module}\\\notag
A &: \text{ module area, $m^2$}\\\notag
I &: \text{ Solar radiation, $W/m^2$ }\\\notag
\tau &: \text{ transmission coefficient of glazing}\\\notag
\alpha &: \text{ absorption coefficient of plate}\\\notag
U_L &: \text{ collector overall heat loss coefficient, $W/m^2$}\\\notag
T_i &: \text{ input fluid temperature, $^\circ C$}\\\notag
T_a &: \text{ average outside air temperature, $^\circ C$}\notag
\end{align}

\begin{figure}[ht]
	\centering
	\includegraphics[width=0.5\textwidth]{phase2/group2/figure/st_module.png}
	\caption{typical module with visualization of calculation parameters (Struckmann (2008))}
	\label{fig:st_module}
\end{figure}

For the calculation of the potential a standard solar thermal module has been taken, the TitanPower-AL2DH Flat Plate Collector from the company SunMaxxSolar \citen{sunmaxx}. The efficiency coefficient is assumed to be $F_R = 0.35$ since this value is also used by SimuPLAN for Solar Atlas Berlin. The input fluid temperature is assumed to be $T_i = 10 ^\circ C$ which seems to be a realistic value for the region of Berlin.  Also the average outside air temperature is taken as  $T_a = 10 ^\circ C$.



\subsection{Validation of Result}
The results are compared and validated against the values computed for the project "Solar Atlas Berlin". According to the roof area which is suitable for solar panels per building, the geometry data source of the Solar Atlas is expected to be of higher quality, because the data has been acquired using a laser scanning system. Therefore, the usable roof area may be exactly predicted for each building. Usable roof area is the area which is not used for any equipment on the roof, such as dormer, chimneys or antennas. The data source used within the GIS Project is an LOD2 Model of Berlin, which does not contain information about roof equipment. For this reason, we use the Solar Atlas Berlin to validate our results.

The value which is validated is calculated potential in MWh/a for photovoltaic systems as well as solar thermal systems. For each building in the test area the difference between the potential given from Solar Atlas Berlin and the potential calculate within the project in calculated. Note, that photovoltaic and solar thermal system are always considered separately.

Out of the differences, the standard deviation can be computed. Since the expectation is known ($e=0$, no difference) the standard deviation is computed as in Equation \ref{eq:validation}.

\begin{align}
\label{eq:validation}
\sigma = \sqrt{\frac{1}{n} \sum\limits_{i=1}^n (diff_i - e )^2}
\end{align}

For the test area a standard deviation of $\sigma_{pv} = 6.45 Mwh/a$ for photovoltaic and $\sigma_{st} = 10.49 Mwh/a$ for solar thermal systems has been reached. A weak spot of this approach is, that outliers influence the result. Although most of the differences are close to zero, the standard deviation is relatively high.
\subsection{Visualization}

To visualize the results a new appearance theme is added using citygml4j. To do this the results have to be classified. For both photovoltaic and solar thermal potential three classes are defined.
\\
Photovoltaic
\begin{itemize}
\item Photovoltaic potential classes
\subitem \(>=50kWh/m^2a\), very good, color: red
\subitem \(<50kWh/m^2a\), good, color: orange
\subitem \(<30kWh/m^2a\), limited, color: yellow
\item Solar thermal potential classes
\subitem \(>=100kWh/m^2a\), very good, color: red
\subitem \(<100kWh/m^2a\), good, color: orange
\subitem \(<50kWh/m^2a\), limited, color: yellow
\end{itemize}

Due to time problems the classification is not optimal. The class boundaries are only empirical values. Furthermore the total energy output depends on the roof area. Which leads to a classification of small roofs to low class although the orientation optimal and the global radiation high. For further investigations this classification should be optimized. The Solar Atlas Berlin uses total global radiation to classify the potential, which could be used as better model.
The new appearance theme is created with citygml4j. Each roof surfaces is added as appearance member. According to the class the are linked with the corresponding diffuse color.
After writing the new CityGML file with all energy related attributes and the new appearance theme. The file is imported to a empty database. Subsequently a KML/Collada export is used to export a KML file with the new appearance IGG\_PV\_Potential. Additionally the address and the values of photovoltaic and solar thermal potential are stored in KML balloons for each building. Figure \ref{fig:vis} shows the resulting KML of the statistical block in Google Earth.

\begin{figure}[hbt!]
\centering
\includegraphics[width=0.7\textwidth]{phase2/group2/figure/viz.png}
\caption{KML export}
\label{fig:vis}
\end{figure}

\subsection{Conclusion}
% TODO: point out weak spots of our implementation and validation (e.g. Solar Atlas is not reliable enough, roof surfaces are not merged)
It can be concluded that the estimation of the photovoltaic potential as well as the solar thermal potential is only applicable to a limited extend. For single buildings the estimation is far to inaccurate, whereas the results for a statistical block become more reliable. The standard deviation for both potentials is too high, it is close to the mean of the potential, which means the result is for a high percentage of all buildings is totally wrong. But the standard deviation is calculated on the basis of the values given by the Solar Atlas Berlin. Therefore the values of the Solar Atlas Berlin are assumed to be correct. That this is not always the case was proved at least with on building. The potential was shifted by one decimal place. Also the geometry data is not sufficient. The used Lod2 (Level of detail 2) geometry only comprises simple polygons to describe the roof surfaces. No geometry of additional roof structures, as dormers, antennas or chimneys are  available. Furthermore the roof surfaces 
representing one roof within the 3D City DB does not always correspond to the real roof surfaces. Some roof surfaces are represented by several small surfaces in the Database. This leads to errors for the estimation of the potentials, because horizontal roof surfaces smaller \(40m^2\) and tilted roof surfaces smaller \(15m^2\) are ignored during the calculation of the potential. \\
The solar potential of roofs depends strongly on the input of solar radiation, which in turn is strongly influenced by shadows. The used shadowing model is very simple. Only shadows due to very high neighboring buildings are considered. Shadows due to additional roof construction are neglected. Tests showed that only a few buildings were neglected due to the implemented shadow model. \\
Nevertheless the estimation of solar potential with existing CityGML data can be very fast and cheap, because no expansive Lidar data is necessary.\\
For the current implementation the disadvantages outbalance the advantages. The results are not accurate enough to use this approach 
\subsection{Future Work}
%TODO: things which may be implemented to improve the results. describe possible solutions to fix the  ``weak spots'' which we pointed %out in the conclusion.

Further investigations to improve the calculation model can lead to better results. Instead of an Lod 2 geometry a higher level of detail as Lod 3 can be used, which describes the geometry of the roofs in more detail. To avoid the neglection of too small roof surfaces due to wrong surface separations, neighboring roof surfaces with equal inclination and orientation should be merged in future.\\
In addition a high order shadow model, which considers partly shadowing of a roof surfaces due to other buildings and shadowing due to additional roof structures. Until now the shadow model is not capable of reducing the global irradiation, but neglects the whole building. The new model should be able to deal with this reduction due to partly shadowed roof surfaces.

%\end{document}

\newpage

\chapter{
\section{Estimation of Residential Electricity Consumption Based on 3D City Model}
}%group 3
\subsection{Define Algorithm}

Normally, the algorithms used to calculate the electricity consumption depend on a large amount of variables and factors, as can be seen on figure 7: \citep{costa2012} \\

\begin{figure}[htb!]
	\centering
	\includegraphics[width=0.8\textwidth]{phase2/group3/fig8.png}
	\caption{Factors for calculating the electricity consumption (Costa, 2012) }
	\label{fig:figure8}
\end{figure}

According to this sketch, in order to calculate the electricity consumption, except the number of people, it is required to know the penetration of appliances, the power of appliances and the hours of use. Such data are changed from one situation to another. The available data cannot support this algorithm, and, therefore, some assumptions are done in order to simplify the process.

According to this situation, a new algorithm is created to estimate the annual consumption based on different household. Figure 9 illustrates the annual consumption for different number of household. Since the number of apartment and residents in each building already calculated in Phase I. If  the residents can be distributed into each apartment as household, the electricity consumption can be calculated from it.

\begin{figure}[H]
	\centering
	\includegraphics[width=0.8\textwidth]{phase2/group3/fig9.png}
	\caption{The annual electricity consumption for different number of household (Vattenfall 2012) }
	\label{fig:figure9}
\end{figure}

\subsection{Data Source}
From Phase I, the results are mainly two indicators: Apartments/volume and Resident/ Building. If these two indicators are applied to all the residential buildings, the number of apartments per buildings will be get. Also the average share of household in Mitte can be found in statistic department.\citep{statsberlin} Combining the electricity consumption data from Vattenfall, all the data for estimating the annual consumption is provided.

\subsection{Method}
Figure 8 illustrates the work flow for calculate the results as well the data sources:

\begin{figure}[htb!]
	\centering
	\includegraphics[width=0.6\textwidth]{phase2/group3/fig10.png}
	\caption{Workflow for Phase II }
	\label{fig:figure10}
\end{figure}

For the last step, there are two ways to calculate the electricity:
1) Establish the regression equation and apply the average household in the equation (see figure 10). This equation was created from the data available on Figure 8.

\begin{figure}[H]
	\centering
	\includegraphics[width=0.6\textwidth]{phase2/group3/fig11.png}
	\caption{The regression equation for electricity consumption}
	\label{fig:figure11}
\end{figure}

2) Distribute the residents into each apartment to get the number of apartment with different household. And use these numbers to multiply the corresponding electricity and lighting consumption. This distribution is base on the statistical share of households for the giving area.

\subsection{Household Distribution}
One of the most important parts in the second method is distributing the inhabitants into apartment. There are three inputs already known: the apartment number, the residents number and the average share of different household. To distribute the residents, only two variables above can be fixed with one variable left. Therefore there are three methods: 

\begin{itemize}
\item Fix the share and residents number, take apartment number as variable
\item Fix the share and apartment number, take residents number as variable
\item Fix the apartment and residents number, take share as variable
\end{itemize}

While for the first two methods the variable parameters will be changed in a unreasonable way, and make the results not reliable, a new distribution method is developed from the third method. 

\subsubsection{Basic Idea}
The basic idea of this method is to try to keep the share close to the average share of Mitte, by applying the following steps:\\
\\
1. Spit the total number of apartment with different household based on the percentage \\
2. Check how many residents left\\
3. Distribute the left residents\\
4. Keep the ratio of the 2~4 household: 3:1:1

\subsubsection{First Distribution}
After the first distribute based on the share, there will be three difference scenes:\\
\begin{itemize}
\item $R > 0$   more people live in one apartment
\item $R < 0$   less people live in one apartment
\item $R < Apartment$  only household(1) and empty apartment
\end{itemize}

In the last scenes, all residents will be put into each apartment with only 1 household and the other apartments are empty. For the first and second scenes, the left residents have to be distributed again.

\subsubsection{Second Distribution}
In the second distribution, as can be seen in figure 11, if only one residents left, it will be put into an apartment with 1 household, which means the number of 1 household will decrease 1 and the number of 2 household will increase 1. If two left, put both of them into 1 household. And the number of 1 household will also decrease 1 but the number of 3 household will increase 1. Do the same as shown in the figure 11 until 8. if there are 9 left, distribute the first 8 as -5:3:1:1 then put the left 1 to 1 household. \\
By this distribution method, it keeps the ratio of the 2~4 household as 3:1:1, which also means every 8 residents as a loop. 


\begin{figure}[H]
	\centering
	\includegraphics[width=0.8\textwidth]{phase2/group3/fig12.png}
	\caption{The second residents’ distribution method (Self Made)}
	\label{fig:figure12}
\end{figure}

\subsection{ArcPython User Interface}

To apply this method to other dataset, an Arctool was designed by using python script which has a good connection with ArcGIS 10. 

\begin{figure}[ht]
	\centering
	\includegraphics[width=0.8\textwidth]{phase2/group3/fig13.png}
	\caption{Developed Arctool (Self Made)}
	\label{fig:figure13}
\end{figure}


\begin{figure}[H]
	\centering
	\includegraphics[width=0.8\textwidth]{phase2/group3/fig14.png}
	\caption{Developed Arctool - State Variables (Self Made)}
	\label{fig:figure14}
\end{figure}

The user can choose the input and output feature class and select the corresponding fields for building age, residents, area and height to calculate the electricity consumption. Besides, the user can also change the optional parameters such as the share of household and the annual electricity consumption for different household.

\subsection{Results}
After the resident distribution, the light and electricity consumption can be computed by multiplying the number of apartment with different household with corresponding consumption

\begin{figure}[ht]
	\centering
	\includegraphics[width=0.7\textwidth]{phase2/group3/fig15.png}
	\caption{Final Household Share (Self Made)}
	\label{fig:figure15}
\end{figure}

\begin{figure}[ht]
	\centering
	\includegraphics[width=1\textwidth]{phase2/group3/fig16.png}
	\caption{Final Lighting Consumption (Self Made)}
	\label{fig:figure16}
\end{figure}

\begin{figure}[H]
	\centering
	\includegraphics[width=1\textwidth]{phase2/group3/fig17.png}
	\caption{Final Electricity Consumption (Self Made)}
	\label{fig:figure17}
\end{figure}
\begin{figure}[H]
	\centering
	\includegraphics[width=0.9\textwidth]{phase2/group3/fig19.png}
	\caption{Final Electricity Consumption per Block (Self Made)}
	\label{fig:figure19}
\end{figure}
Table 2 shows the results from the two calculation methods. As can be seen from it, using the average household and regression equation the results are smaller than the distribution method. Nevertheless, both estimation methods rendered values within a short interval - as both are based essentially on the residents per building and/or per apartment. 


\begin{table}[H]
	\centering
	\includegraphics[width=0.8\textwidth]{phase2/group3/fig18.PNG}
	\caption{Final Electricity Consumption (Self Made)}
	\label{fig:figure18}
\end{table}



\subsection{Conclusion}
The described method is able to integrate statistical data, surveying data and virtual 3D City Models. Furthermore, it provides a rough tool to calculate the electricity for other area which can be implemented in 3D City Models. As a result, instead of having only average results, the calculations are also locally performed. In other words, as the parameters are first calculated per each building, it is also possible to select the final results for each building. The two distribution methods provide the user two different level results with different data. If there is no household share data, the regression equation method can give the rough results. While with the household share data, the distribution method can be applied for more reasonable results.\\

The 3D Model is able to provide all the demanded geometrical factors. Once the electricity consumption estimation is geometrically based on the volume of the buildings, data regarding its footprint area and height are required. \\

However, there still some parts need to pay attention to during applying this method. Such as the function of the building, the relationship between 3D City Model and the survey address.



\pagebreak
\bibliographystyle{plain}
\bibliography{references}{}
\end{document}
