\subsection{Validation of Result}
The results are compared and validated against the values computed for the project "Solar Atlas Berlin". According to the roof area which is suitable for solar panels per building, the geometry data source of the Solar Atlas is expected to be of higher quality, because the data has been acquired using a laser scanning system. Therefore, the usable roof area may be exactly predicted for each building. Usable roof area is the area which is not used for any equipment on the roof, such as dormer, chimneys or antennas. The data source used within the GIS Project is an LOD2 Model of Berlin, which does not contain information about roof equipment. For this reason, we use the Solar Atlas Berlin to validate our results.

The value which is validated is calculated potential in MWh/a for photovoltaic systems as well as solar thermal systems. For each building in the test area the difference between the potential given from Solar Atlas Berlin and the potential calculate within the project in calculated. Note, that photovoltaic and solar thermal system are always considered separately.

Out of the differences, the standard deviation can be computed. Since the expectation is known ($e=0$, no difference) the standard deviation is computed as in Equation \ref{eq:validation}.

\begin{align}
\label{eq:validation}
\sigma = \sqrt{\frac{1}{n} \sum\limits_{i=1}^n (diff_i - e )^2}
\end{align}

For the test area a standard deviation of $\sigma_{pv} = 6.45 Mwh/a$ for photovoltaic and $\sigma_{st} = 10.49 Mwh/a$ for solar thermal systems has been reached. A weak spot of this approach is, that outliers influence the result. Although most of the differences are close to zero, the standard deviation is relatively high.