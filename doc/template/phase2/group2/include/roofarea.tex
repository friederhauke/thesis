\subsection{Reduction of Roof Surface Area due to Roof Equipment}
\label{sec:roofarea}
The main problem when using a LOD 2 model is that the suitable roof area is not known. Fact is, that not 100\% of the roof can be used to install solar panels, since roofs may have dormers, antennas or chimneys which are not part of the LOD2 model. But the roof equipment is considered for the Solar Atlas Berlin and with use of this data source empirical reduction factors may be computed. Therefore group 2 implemented a program, which reads a cityGML file, containing all buildings of the test area (~1000 buildings) and computes the reduction factor as in Equation \ref{eq:roof}.

\begin{align}
\label{eq:roof}
r_E &= \frac{A_{SAB}}{A_{LOD2}} \\\notag\\\notag
\text{with:}&\\\notag
r_E &:\text{ empirical reduction factor}\\\notag
A_{SAB} &: \text{Roof Area according to solar atlas Berlin }\\\notag
A_{LOD2} &: \text{ roof area calculated from LOD2 model in 3D CityDB }\\\notag
\end{align}

The Solar Atlas Berlin provides different areas for photovoltaic and solar thermal systems. Therefore also different reduction factors are computed. The reduction factors are also separated between exclusively flat roof and mixed roofs, contain flat as well as tilted surfaces. This is important because flat roofs are more likely equipped. Additionally, the age class of the building has been taken into account. Tables \ref{tab:roofarea_st} and \ref{tab:roofarea_pv} show the result, including the number of surface and the variance of the value. It can be seen that, some reduction factors are not representative, because not enough roofs of this type are in the test area. With the use of a database of entire Berlin might fix the Problem.

 

\begin{table}
\centering 

\begin{tabular}{|c||c|c|c||c|c|c|}
  \hline
  \multirow{2}{*}{Age Class} & \multicolumn{3}{|c||}{flat roof} &  \multicolumn{3}{|c|}{mixed roof}\\
  \cline{2-7}
  & $r_E$ & count & $\sigma^2$  & $r_E$ & count & $\sigma^2$\\
  \hline
1899&0.198&94&0.016&0.440&199&0.056\\\hline
1918&0.152&56&0.019&0.435&233&0.059\\\hline
1932&0.167&13&0.017&0.506&9&0.048\\\hline
1945&0.304&2&0.0008&0.765&1&0.000\\\hline
1961&0.221&74&0.013&0.581&56&0.049\\\hline
1974&0.198&90&0.015&0.371&17&0.080\\\hline
1993&0.219&66&0.006&0.293&5&0.016\\\hline
2012&0.149&91&0.019&0.487&21&0.165\\\hline
\end{tabular}
\caption{empirical reduction factors for calculation of energy gain using solar thermal collectors }
 \label{tab:roofarea_st}
\end{table}

\begin{table}
\centering 
\begin{tabular}{|c||c|c|c||c|c|c|}
  \hline
  \multirow{2}{*}{Age Class} & \multicolumn{3}{|c||}{flat roof} &  \multicolumn{3}{|c|}{mixed roof}\\
  \cline{2-7}
  & $r_E$ & count & $\sigma^2$  & $r_E$ & count & $\sigma^2$\\
  \hline
1899&0.127&94&0.019&0.349&199&0.050\\\hline
1918&0.097&56&0.016&0.323&233&0.053\\\hline
1932&0.056&13&0.008&0.403&9&0.076\\\hline
1945&0.000&2&0.000&0.730&1&0.000\\\hline
1961&0.147&74&0.015&0.498&56&0.050\\\hline
1974&0.129&90&0.016&0.289&17&0.081\\\hline
1993&0.128&66&0.011&0.146&5&0.013\\\hline
2012&0.090&91&0.016&0.399&21&0.147\\\hline
\end{tabular}
\caption{empirical reduction factors for calculation of energy gain using photovoltaic modules }
 \label{tab:roofarea_pv}
\end{table}

