\subsection{Conclusion}
% TODO: point out weak spots of our implementation and validation (e.g. Solar Atlas is not reliable enough, roof surfaces are not merged)
It can be concluded that the estimation of the photovoltaic potential as well as the solar thermal potential is only applicable to a limited extend. For single buildings the estimation is far to inaccurate, whereas the results for a statistical block become more reliable. The standard deviation for both potentials is too high, it is close to the mean of the potential, which means the result is for a high percentage of all buildings is totally wrong. But the standard deviation is calculated on the basis of the values given by the Solar Atlas Berlin. Therefore the values of the Solar Atlas Berlin are assumed to be correct. That this is not always the case was proved at least with on building. The potential was shifted by one decimal place. Also the geometry data is not sufficient. The used Lod2 (Level of detail 2) geometry only comprises simple polygons to describe the roof surfaces. No geometry of additional roof structures, as dormers, antennas or chimneys are  available. Furthermore the roof surfaces 
representing one roof within the 3D City DB does not always correspond to the real roof surfaces. Some roof surfaces are represented by several small surfaces in the Database. This leads to errors for the estimation of the potentials, because horizontal roof surfaces smaller \(40m^2\) and tilted roof surfaces smaller \(15m^2\) are ignored during the calculation of the potential. \\
The solar potential of roofs depends strongly on the input of solar radiation, which in turn is strongly influenced by shadows. The used shadowing model is very simple. Only shadows due to very high neighboring buildings are considered. Shadows due to additional roof construction are neglected. Tests showed that only a few buildings were neglected due to the implemented shadow model. \\
Nevertheless the estimation of solar potential with existing CityGML data can be very fast and cheap, because no expansive Lidar data is necessary.\\
For the current implementation the disadvantages outbalance the advantages. The results are not accurate enough to use this approach 